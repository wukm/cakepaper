\documentclass[12pt]{csulb-thesis-old}
%\usepackage{csulbthm} % this is broken. need to fix and use instead of amsthm?
\usepackage{amsthm}
\usepackage{setspace}
\usepackage{pslatex}
\usepackage{amsfonts, amsmath, mathrsfs}
\usepackage{amsxtra, amssymb, amscd, mathtools}
\usepackage{graphicx}
\usepackage{subfig}
\usepackage{rotating}
\usepackage{bm}
\usepackage{hyperref}
\usepackage[capitalise]{cleveref}

\usepackage{multicol}
\usepackage{lscape}
\usepackage{datatool} % Allows importing tables
\usepackage{longtable}
\usepackage[inline]{enumitem}
\usepackage[dvipsnames]{xcolor}
\usepackage[normalem]{ulem}
\usepackage{minitoc}
%\usepackage{booktabs} % requires for pandas.DataTable.to_latex() for listings?
\usepackage{xspace} %punctuation sensitive spacing at end of ensuremath environment
\usepackage{tabulary} %auto word wrapping on tables

\usepackage[T1]{fontenc}
\usepackage{newtxtext,newtxmath} 
\usepackage[utf8]{inputenc}



%%%%%%% VISIBLE COMMENTS / TODOS / CLEANUP TODOS %%%%%%%%%%%%%%%%%%%%%%%%%%%%%%%%%%%%%%%%%%%%%%%%%%%%
%%%%%%%%%%%%%%%%%%%%%%%%%%%%%%%%%%%%%%%%%%%%%%%%%%%%

% to make interspersed comments visible, uncomment these three lines and comment out the three below
%\newcommand{\vcomment}[1]{\textit{\textcolor{Blue}{[#1]}}}
%\newcommand{\vtodo}[1]{\textbf{\textcolor{Red}{[TODO:\;#1]}}}
%\newcommand{\vcleanup}[1]{\textcolor{Orchid}{[CLEANUP:\;#1]}}

% to hide interspersed comments, uncomment these three lines and comment out the three above
\newcommand{\vcomment}[1]{}
\newcommand{\vtodo}[1]{}
\newcommand{\vcleanup}[1]{}

%%%%%%%%%%%%%%%%%%%%%%%%%%%%%%%%%%%%%%%%%%%%%%%%%%%%%%%%%%%%%%%%%%%%%%%%%%%%%%%%%%%%%%%%%%

\usepackage{color}

\definecolor{mygreen}{rgb}{0,0.6,0}
\definecolor{graygreen}{rgb}{0.23,0.33,0.25}
\definecolor{mymauve}{rgb}{0.58,0,0.22}
\definecolor{mygray}{rgb}{0.32,0.32,0.32}


\usepackage{listings}

%\usepackage{pxfonts}
\newcommand*\lstinputpath[1]{\lstset{inputpath=#1}}

\lstinputpath{listings}
\lstset{ 
	%inputencoding=utf8,
	extendedchars=false,
	mathescape=true,
	backgroundcolor=\color{white},   % choose the background color; you must add \usepackage{color} or \usepackage{xcolor}; should come as last argument
	basicstyle=\linespread{0.8}\ttfamily\footnotesize,   % the size of the fonts that are used for the code
	breakatwhitespace=false,         % sets if automatic breaks should only happen at whitespace
	breaklines=false,                 % sets automatic line breaking
	captionpos=t,                    % sets the caption-position to bottom
	commentstyle=\color{graygreen},    % comment style
	deletekeywords={...},            % if you want to delete keywords from the given language
	escapeinside={\%@}{@\%)},          % if you want to add LaTeX within your code
	%extendedchars=true,              % lets you use non-ASCII characters; for 8-bits encodings only, does not work with UTF-8
	frame=bt,	                   % adds a frame around the code
	keepspaces=true,                 % keeps spaces in text, useful for keeping indentation of code (possibly needs columns=flexible)
	keywordstyle=\bfseries\color{blue},       % keyword style
	morekeywords={*,...},            % if you want to add more keywords to the set
	numbers=left,                    % where to put the line-numbers; possible values are (none, left, right)
	numbersep=5pt,                   % how far the line-numbers are from the code
	numberstyle=\tiny\color{mygray}, % the style that is used for the line-numbers
	rulecolor=\color{black},         % if not set, the frame-color may be changed on line-breaks within not-black text (e.g. comments (green here))
	showspaces=false,                % show spaces everywhere adding particular underscores; it overrides 'showstringspaces'
	showstringspaces=false,          % underline spaces within strings only
	showtabs=false,                  % show tabs within strings adding particular underscores
	stepnumber=1,                    % the step between two line-numbers. If it's 1, each line will be numbered
	stringstyle=\color{mymauve},     % string literal style
	tabsize=4,	                   % sets default tabsize to 2 spaces
	title=\lstname                   % show the filename of files included with \lstinputlisting; also try caption instead of title
} %end lstset

%%%%%%%%%%%%%%%%%%%%%%%%%%%%%%%%%%%%%%%%%%%%%%%%%%%%%%%%%%%%%%%%%%%%%%%%%%%%%%%%%

% math commands

% mathy definitions?
\newtheorem{defn}{Definition}[chapter]
\newtheorem{theorem}{Theorem}[chapter]
\newtheorem{corollary}{Corollary}[theorem]
\newtheorem{lemma}[theorem]{Lemma}
\newtheorem{axiom}{Axiom}[chapter]


\newcommand{\R}{\mathbb{R}} % the standard 'set of all real numbers'
\newcommand{\N}{\mathbb{N}} % set of all natural numbers
\newcommand{\Z}{\mathbb{Z}} % integers
\newcommand{\Zpos}{\mathbb{Z}_{+}}
\newcommand{\image}{\mathrm{image}} % image of a function
\newcommand{\spn}{\mathrm{span}}

\newcommand{\Vmax}{\ensuremath{\mathcal{V}_{\max}}\xspace}
\newcommand{\VSigma}{\ensuremath{\mathcal{V}_{\Sigma}}\xspace}
\newcommand{\Vsigma}{\ensuremath{\mathcal{V}_{\sigma}}\xspace}
\newcommand{\Sigmaneg}{\Sigma^{(-)}}	

\newcommand{\VSigmapos}{\ensuremath{\mathcal{V}_{\Sigma}^{(+)}}\xspace}
\newcommand{\VSigmaneg}{\ensuremath{\mathcal{V}_{\Sigma}^{(-)}}\xspace}

\newcommand{\Vmaxpos}{\ensuremath{\mathcal{V}^{(+)}_{\max}}\xspace}
\newcommand{\Vmaxneg}{\ensuremath{\mathcal{V}^{(-)}_{\max}}\xspace}
\newcommand{\alphapos}{\alpha^{(+)}}
\newcommand{\alphaneg}{\alpha^{(-)}}
\newcommand{\alphaiso}{\alpha_{\small{\textsc{ISO}}}}


\newcommand{\Hess}{\mathrm{Hess}} % fancy H for the hessian
\newcommand{\wein}{\mathsf{L}} % for the Weingarten map
\newcommand{\weinmat}{\widehat{\mathsf{L}}}
\newcommand{\img}{\mathtt{I}} % discrete image (a matrix, not the math term)

\DeclarePairedDelimiterX{\inner}[2]{\langle}{\rangle}{#1, #2}
\DeclarePairedDelimiterX{\vnorm}[1]{\Vert}{\Vert}{#1}
\DeclarePairedDelimiterX{\abs}[1]{\vert}{\vert}{#1}

\newcommand{\bigO}{\mathcal{O}}

\DeclareMathOperator{\FT}{\mathcal{F}} % fancy F for \FT
\DeclareMathOperator{\DFT}{\mathcal{D}} % fancy D for \DFT


%\newcommand*\mcol[1]{\overset{\big\uparrow}{\underset{\big\downarrow}{#1}}}
\newcommand*\mcol[1]{\overset{\big\vert}{\underset{\big\vert}{#1}}}
%\usepackage[small,singlelinecheck=on]{caption}
%\usepackage[font=singlespacing]{caption}

% each subfolder should be added here. also must end in a /


\graphicspath{{figures/}
			  {figures/frangi3d/}
			  {figures/strict_examples/}
			  {figures/inset_of_sample_issues/}
			  {figures/rw_demo/}
			  {figures/output_examples/}
		  	  {figures/qthresh_demo/}
		  	  {figures/segmentation_montage/}
        	  {figures/segmentation/}
		  	}


% this fixes stuff in the style file I guess? it's illegible
\makeatletter
\def\LT@c@ption#1[#2]#3{%
  \LT@makecaption#1\fnum@table{#3}%
  \def\@tempa{#2}%
  \ifx\@tempa\@empty\else
     {\vspace*{\baselineskip}
     \let\space
     \addcontentsline{lot}{table}{\protect\numberline{4.}{#2}}}%
  \fi}
\makeatother


%%%%%%%%%%%%%%%%%%%%%%%%%%About the Author%%%%
\lastdegree{B.S.} 							%B.S., B.A., etc.
\lastdegyear{2013} 							% year of last degree
\lastdeginst{University of California, Los Angeles}	% University granting degree listed above
\degree{Master of Science in Applied Mathematics}				%CSULB degree
\department{Mathematics and Statistics} 			% department
\gradyear{2019} 							% year you will get current degree
\gradmonth{May} 							% graduation month for current degree
\author{Lucas Allen Wukmer}  					                %Your official CSULB name.

%Enter the title in all caps force returns using \\ to form an inverted pyramid
\title{OPTIMIZED STRICT MULTISCALE FRANGI PREFILTERING FOR SEGMENTATION:
	     TOWARDS AN AUTOMATED PLACENTAL CHORIONIC SURFACE VASCULAR NETWORK EXTRACTION}

%%%%%%%%%%%%%%%%%%%%%%%%%Thesis Committe%%%%%%%%%%%%%%%%%%%%%%%%%%%%%%%%%%%%%%%%%%%%%%%%%%%
\committeeA{Jen-Mei~Chang,~Ph.D.~(Chair)} 						% insert name of thesis chair; include his/her degree
\committeeAdept{Mathematics and Statistics} 
\committeeB{James~von~Brecht,~Ph.D.} 							% name and degree of 2nd committee member
\committeeBdept{Mathematics and Statistics}
\committeeC{William~Ziemer,~Ph.D.} 							% name and degree of 3rd committee member
\committeeCdept{Mathematics and Statistics} 
\deptHead{Tangan~Gao,~Ph.D.} 						        % name and degree of department chair
\deptHeadTitle{Department Chair, Mathematics and Statistics} 


\begin{document}
\begin{preliminary}

% According to the 2019 guidelines title page comes first before abstract
% then abstract starts at (ii). Also the signature page isn't part of the
% formal document anymore. It's electronic?

%% There should be NOTHING between the title page and abstract.
%% This generates the title page from the information given above.
\maketitle

%% This generates the abstract page, with the line spacing adjusted
%% according to University guidelines.
%%% The university says(At most 150 words or 2 pages).
%% Use \input{} instead of \include{} because \include{} will add a page break and \input{} will not
\begin{abstract}
	%%%%% Abstract

% University Guidelines suggest less than 150 words or 2 pages

Recent statistical analysis of placental features has suggested the usefulness
of studying key features of the placental chorionic surface vascular network
(PCSVN) as a measure of overall neonatal health \cite{chang2017}. A recent
study has suggested that reliable reporting of these features may be useful in
identifying risks of certain neurodevelopmental disorders at birth. The
necessary features can be extracted from an accurate tracing of the surface
vascular network, but such tracings must still be done manually, with
significant user intervention. Automating this procedure would not only allow
more data acquisition to study the potential effects of placental development on
later conditions, but even perhaps provide a real-time diagnostic for neonatal
risk factors.

Much work has been to develop reliable vascular network segmentation methods for
well-known image domains (such as retinal MRA images) using Hessian-based
filters, namely the multiscale Frangi filter. It is desirable to extend these
techniques to study placental images, but the approach has been historically
hindered by the comparative irregularity of the placental surface as a whole,
which introduces significant noise into filtered result.  Previous work has either
involved additional filtering \cite{huynh2013filter} or other techniques that are
often time-consuming and resource intensive \cite{djima2017enhancing}.

Here we provide an in-depth mathematical background of the multiscale Frangi filter.
Informed by this theory, we are able to identify stricter parameters that allow us
to greatly improve our result. We also reimplement the Frangi filter in frequency space
(using a fast fourier transform), which allows us to quickly probe many scales.

We demonstrate the effectiveness of our sped-up implementation of the Frangi
filter by performing a large (N=20) multiscale Frangi filter on a set of 201
placental images from a private database provided by the National Children's
Study (NCS). We then compare several approaches of merging the multiscale
result into an approximation of the PCSVN and compare them to manual tracings
of the network. Finally, we develop the notion of the \textit{signed} Frangi filter,
upon which we describe a novel-yet-straightforward segmentation method called "trough filling".



\end{abstract}

%% This generates an "acknowledgements" section, if needed. Comment out if not needed.
%% Use \input{} instead of \include{} because \include{} will add a page break and \input{} will not
\begin{acknowledgements}
%%acknowledgements.tex

Thank you to people for things.

Thank you for reading this.
\end{acknowledgements}

%% This generates the Table of Contents (on a separate page).
\setcounter{tocdepth}{0}  %%  Set how many levels to include in the TOC
\dominitoc
\tableofcontents

%% tocdepth counter affects list of tables and figures too.
%% they'll appear blank if tocdepth is 0 above
\setcounter{tocdepth}{1}
%% This generates the List of Tables (on a separate page), if needed.
%% (uncomment to have it appear in the document)
\listoftables

%% This generates the List of Figures (on a separate page), if needed.
%% (uncomment to have it appear in the document)
\listoffigures

%% End of the preliminary sections: reset page style and numbering.
\end{preliminary}

\clearpage 
\pagenumbering{arabic}	%Starts numbering pages in arabic, starts from 1.

%% introduction - describe the motivation and how previously published
%% on the topic integrates with the current work

\chapter{Introduction}

	\section{The Applied Problem}
		Reference Nen's paper and latest autism risk paper.
	\section{Context in Image Processing}

		\begin{itemize}
			\item brief background of math image processing methods
			\item what's been tried in this applied problem
			\begin{itemize}
				\item nen \cite{huynh2013filter}
				\item catalina's paper
				\item kara's paper
				\item other domains
			\end{itemize}
		\end{itemize}
	\section{Research Goals}
	Segue from previous paragraph, talk about strengths and weaknesses of other methods and what this research aims to accomplish. Include `research questions' that could allow a reader to answer the question "will this research work for my problem?". "Elevator pitch" maybe goes here.
	\section{Roadmap}
	Outline  of the thesis ("firstly" bullshit)		
\chapter{Differential Geometry and Surface Curvature} \label{ch:diffgeo}


Our goal is establish a resource efficient method of finding curvilinear content in 2D grayscale digital images using concepts of differential geometry. We proceed by
\begin{enumerate*}[label=(\roman*)]
	\item establishing a standard method of viewing these images as 2D surfaces,
	\item developing a minimal yet rigorous distillation of differential geometry
			to obtain suitiable quantifiers
			for the study of curvilinear structure in 3D surfaces,
	\item establishing a filter based on these quantifiers,
	and finally
	\item developing methods necessary for efficient computation of the filter.
\end{enumerate*}


%%%%%%%%%%%%%%%%%%%%%%%%%%%%%%%%%%%%%%%%%%%%%%%%%%%%%%%
\section{Problem Setup in Image Processing}\label{sec:image-processing-setup}

%\vcomment{This section should be basically describing how to view an image as a surface.
%  Use any notation here that is useful beyond differential geometry, in all the contexts
%  we need to consider the image (in terms of Fourier Theory, scale space theory,
%  Frangi filtering, etc. all together.}
A digital 2D grayscale image is given by a $M\times N$ array of pixels, whose intensity is given by an integer value between 0 and 255.

\begin{defn}[Image as a pixel matrix] \label{def:image_as_pixel_matrix}
	\begin{equation*}
	\img \in \N^{M \times N}
	\quad\textrm{with}\quad
	0 \le \img_{ij} \le 2^8 - 1
	\end{equation*}
\end{defn}
	% iI hate this wording.
	For theoretical purposes, we wish to consider any such picture to ultimately be a sampling of a 2D continuous surface. We also require that this surface is sufficiently continuous as to admit the existence of second partial derivatives.
	
\begin{defn}[Image as an interpolated surface] \label{def:image_as_surface}
 \begin{equation*}
 h: \R^2 \to \R
 \quad \textrm{with}\quad
 h \in \mathcal{C}^2\left(\R^2\right),
 \quad\textrm{where}\quad
    h(i,j) = \mathtt{I}_{ij}
    \; \forall (i,j) \in
     \left\{0,...,M\right\} \times
     \left\{0,...,N\right\} \subset \N^2
    \end{equation*}
\end{defn}
% say the same thing with words
That is, the function $h$ is identical to the pixel matrix $\img$ at all integer inputs,
and simply a ``smooth enough'' interpolation of those points for all other values.


It is of course necessary to admit that $\mathtt{I}$ is not really a perfect representation of the underlying ``content'' within the picture. Not only is information lost when $\img$ is stored as an integer, there are also elements of noise and anomalies of lighting that would constitute noise to the original signal. There are multiple treatments of image processing that do address this discrepancy in a pragmatic way \cite{DIPGW}, especially when the goal is noise reduction.
However, we will be content to simply represent the pixels of $\img$ as the ultimate ``cause'' of the surface $h$ in \cref{def:image_as_surface}, and worry not about how faithfully that sampling corresponds to the real world.
% Word salad, also rude? ask chang about this
Moreover, though our samples in the image domain have been carefully prepared (as outlined in \cref{sec:NCS-data-set}), there are numerous shortcomings therein, and improvements to the veracity of our original signal could be made from many angles.
Though we shall draw upon the notion of the pixel matrix $\img$ as a sampling again to motivate our development of scale space theory in \cref{sec:scale-space-theory}, we ultimately use these techniques because we find them successful to our problem.
 
%%%%%%%%%%%%%%%%%%%%%%%%%%%%%%%%%%%%%%%%%%%%%%%%%%%%%%%
%%%%%%%%%%%%%%%%%%%%%%%%%%%%%%%%%%%%%%%%%%%%%%%%%%%%%%%
\section{Differential Geometry} \label{sec:differential-geometry}
  
We wish to describe the structure of an image as a surface. To do this, we develop the notion of curvature of a surface in $\R^3$ in a standard way \cite{Kuhnel-DiffGeo}.

\subsection{Preliminaries of Differential Geometry}
%\vcleanup{Mention when you're talking about a general surface in $\R^3$ and when it's specifically a graph and make sure it's clear which case you're dealing with and motivate why you'd want to talk about things that aren't graphs at all (to define shape operator in general). Potentially rework to not talk about non-graphs when irrelevant. Although it's useful to keep some references to non-graphs to align this with "the general picture" and what you'll find in a diff geo textbook, also because I already wasted time doing the hard work.}
    Given an open subset $U\subset \R^2$ and a twice differentiable function  $h: U \to \R$ (as in \cref{def:image_as_surface})
    we define the graph, $f$, of $h$ in the following definition.
    
    \begin{defn} \label{def:graph}
    The surface $f$ is a graph (of the function $h$) when 
    \[
     f: U \to \R^3 \quad \textrm{by} \quad f(u_1, u_2) \;=\; \big(u_1 , u_2 , h(u_1,u_2)\big)
     \;,\quad u = (u_1, u_2) \in U \subset \R^2 \]
    \end{defn}
    %That is, $f$ is an embedding/immersion/parametrization of the function h
    Since the graph $f$ is clearly one-to-one (into $\R^3$) by definition, we may readily associate any input $u\in U$ with
    its corresponding output $p \in f[U]$, i.e.
    $ p = f(u) = f(u_1, u_2) \;=\; \big(u_1 , u_2 , h(u_1,u_2)\big)$,
    % world salad but may need motivation.
    depending on whether we wish to focus on a point of a graph in terms of its input
    or in terms of the structure of the graph itself.
    
    % Motivation
    Our development of curvature ultimately will hinge upon a careful consideration of the tangent plane of $f$ at a point $p$, for we will require a concrete definition of both the tangent space within the domain and image of $f$,
    as well as the differential" of $f$,
    % shit or not because we need it later!
    % the lattermost of which we will only define for the immediate case required.
	%    Seeing that $f$ is one-to-one should make a lot of this
	%    futzing about complete overkill, but I've yet to find a way to distill it.
	%    That is, this development works for any parametrized surface element, not necessarily a graph. Whatever for now.
    
    
    \begin{defn}[Tangent space of $U$ at $u$] \label{def:tangent-at-U}
    	\[ T_{u} U = \left\{u\right\} \times \R^2
    	\]
    	\end{defn}
    \begin{defn}[Tangent space of $\R^3$ at $p$] \label{def:tangent-of-R3}
    	\[ T_{p} \R^3 = \{p\} \times \R^3
    	\]
    \end{defn}
    It is immediately clear that $T_uU$ and $T_p\R^3$ are isomorphic to
    $\R^2$ and $\R^3$, respectively, and we can easily visualize elements of $T_uU$ are tangent vectors in $\R^2$ ``originating'' at the point $u$, and elements of $T_p\R^3$ are tangent vectors ``originating'' at the point $p$.
\begin{defn}[The differential of $f$ at a point $u$] \label{def:differential-map}
       	$Df\vert_u$ is the map from $T_uU$ into $\R^3$ given by
    \[
     Df|_u : T_uU \to T_{f(u)}\R^3
     \quad \textrm{by}
     \quad w \mapsto J_f (u) w
    \]
where $J_f(u)$ is the Jacobian of $f$ evaluated at some fixed point $u \in U$, i.e. the matrix
\[
J_f (u) = \left[ \left.\frac{\partial f_i}{\partial u_j}\right\vert_u \right]_{i,j}
\]
\end{defn} % You can in some ways argue that J_x(f) is a better notation
%Although not necessary presently, we could just as easily consider the differential of an arbitrary function as a map between tangent vectors in the function's domain and tangent vectors in its range.
We could also just identify the differential \cref{def:differential-map}  with a mapping $U \to \R^3$ by the obvious isomorphism inherent in \cref{def:tangent-at-U}. In this case, the differential of $f$ at $x$ is simply a linear transformation between the tangent spaces $T_uU$ and $T_p\R^3$ where the matrix of the transformation is given by the Jacobian. We can define such a differential at any point $u$ in the domain.

With these three definitions, we are equipped to give a formal definition of $T_uf$,
the tangent plane of $f$ at an input $u$.
\begin{defn}[Tangent plane of a graph]\label{def:tangent-plane}
\[
T_u f := Df\vert_u \left(T_u U\right)
\subset T_{f(u)} \R^3 = T_{p} \R^3
\]
\end{defn}

The vectors in this plane can thus be identified as tangent vectors from $T_uU$ that have been passed through the differential mapping $Df\vert_u$.
We shall denote a generic tangent vector $X \in T_u f$ at point $p$.
%%%%%STOPPEED HERE
We may expand any such vector $X$ in terms of the basis $\left\{ \frac{\partial f}{\partial u_i}\right\}_{i=1,2}$; that is,
$\textrm{span}\left\{ \frac{\partial f}{\partial u_1}, \frac{\partial f}{\partial u_2}\right\} = T_u f$. Each of these basis vectors are the columns of the Jacobian matrix and are \textit{parametric} derivatives of $f$ for the input $u$.  

% level of "needless" abstraction
Given the level of abstraction above, it may be refreshing to explicitly show the linear independence of this set in the case of an arbitrary graph $f$.
%it's the span though? just say it's the span?
\begin{lemma} \label{lemma:f_ui-is-a-basis}
	When $f$ is a graph, for all points $u \in U$, $\left\{\frac{\partial f}{\partial u_1} , \frac{\partial f}{\partial u_2}\right\}$ is in fact a basis for the tangent plane $T_uf$.
\end{lemma}

\begin{proof}
Given the definition of a graph $f$ as in \cref{def:graph}, we can directly calculate the parametric derivatives of $f$ at a point $u$.
\[
f_{u_1} = (1,0,h_{u_1}(u)) \quad\textrm{and}\quad f_{u_2} = (0,1,h_{u_2}(u))
\]
 which are obviously linearly independent.  Then $Df\vert_u (1,0) = f_{u_1} ,$ and $ Df\vert_u (0,1) = f_{u_2}$, which shows $\left\{\frac{\partial f}{\partial u_1} , \frac{\partial f}{\partial u_2}\right\} \in T_uf$.  Thus $\left\{\frac{\partial f}{\partial u_1} , \frac{\partial f}{\partial u_2}\right\}$ is a linearly independent subset of $T_u f$, and can serve as its basis.\end{proof}

Quite obviously, we're assuming $(1,0), (0,1) \in U$. If this is not the case, we pick some $\alpha$ small enough so that $(\alpha,0)$ and $(0,\alpha)$ are contained and this scaled version would serve as a basis instead.

	The parametric derivatives of $f$ are not, in general, orthogonal at any point $u$, unless it happens that $h_{u_1} $ or $h_{u_2}$ is zero.
	A visualization of some of the above is given in \cref{fig:Tuf}, although note that $f_{u_1}$ and  $f_{u_2}$ accidentally appear orthogonal.
	
	\begin{figure}\centering
		\includegraphics[width=0.8\textwidth]{T_uf}
		\caption{Tangent plane of a graph}
		\label{fig:Tuf} 
	\end{figure}
%  	But the above allows us to view $Df\vert_u$ differential map of $f$ is exactly the expansion of
%  	a point  $v \in U$ along the basis
%  	$\left\{\frac{\partial f}{\partial u_1} , \frac{\partial f}{\partial u_2}\right\}$ .

We now concern ourselves with developing the notion of curvature on a surface. First, we need to consider an arbitrary regular curve (i.e. differentiable, one-to-one, non-zero derivative) contained within the image of $f$. 
  	
  	% this isn't really a definition...?
 

\subsection{Curvature of a surface and its calculation}
In the context of a regular arc-length parametrized curve $c: I \to \R^3$ parametrized along some closed interval $I\in \R$
 (that is, a differentiable, one-to-one curve where $c'(s) = 1 \;\; \forall s \in I$), curvature at a point $s \in I$ is defined simply as the magnitude of the curve's acceleration: $\kappa(s) := \vnorm{c''(s)}$.
 
 To extend the notion of curvature of a surface $f$, we can consider the curvature of such an arbitrary curve embedded within the surface.
 
 \begin{defn}[Surface curve] \label{def:curve-on-a-surface}
 	Given a closed interval $I \subset \R$, we call the regular curve
 	$c: I \to \R^3$ a surface curve in the event that $\image(c) \subset \image(f)$ entirely. The one-to-one-ness of the graph $f$ ensures that we can define (for the given curve) an intermediary parametrization $\theta$  so that
 	$ c = f \circ \theta $. That is,
 	\[
 	\theta : I \to U \; \textrm{by} \; \theta(t) = \big(\theta_1(t), \theta_2(t)\big)
 	\]
 	so that $c(t) = f(\theta_c(t)) \;\forall t\in I,$
 	and $c[I] = f\left[\theta_c[I]\right]$.
 \end{defn}
 Note as well that the velocity of this particular curve lies within $T_u f$. This
 can be seen by an elementary application of chain rule:
 \begin{align}
 \frac{d c}{dt} &= \frac{d}{dt}\big[ f(\theta_c(t))\big] \nonumber \\
 &= \frac{d}{dt}\big[f(\theta_1(t), \theta_2(t))\big] \nonumber \\
 &= \theta_1'(t)\left( \frac{\partial f}{\partial u_1} \right) + 
 \theta_2'(t)\left( \frac{\partial f}{\partial u_2} \right) \in T_uf.
 \end{align}
 
 Considering the parameter $p \in I$ and its associated point $u = \theta_c(p)$, we wish to compare the curvatures of all (regular) surface curves passing through the point $u$ at some particular velocity.
 
 We now present a main result that provides a notion of curvature of a surface.

	\begin{theorem}[Theorem of Meusnier] \label{thm:meusnier}
		Given a point $u \in U $ and a tangent direction $X \in T_u f$,
  any regular curve on the surface $c: I \to \image(f)$ with $p\in I : \theta_c(p) = u$
  where $c'(p) = X$ will have the same curvature.
	\end{theorem}
	
	%\vtodo{provide a visualization of this}
	
	In other words, any two curves on the surface with a common velocity at a given point on the surface will have the same curvature. To prove this, we require one final definition.
	
	\begin{defn}[The Gauss Map] \label{def:gauss-map}
		The Gauss map at a point $u$  is the unit normal to the tangent plane
		\[\nu : U \to \R^3 \quad\textrm{by}\quad  \nu(u) :=
		\frac{\frac{\partial f}{\partial u_1} \times \frac{\partial f}{\partial u_2}}
		{\vnorm{\frac{\partial f}{\partial u_1} \times \frac{\partial f}{\partial u_2}}} \]
	\end{defn}
	Each partial above understood to be evaluated at the input $u \in U$; that is, we calculate $\left.\frac{\partial f}{\partial u_i}\right|_u$.
	The existence of the cross product in its definition makes it clear that $\nu \perp \frac{\partial f}{\partial u_i}$ each $i=1,2$. A simple dimensionality argument of $\R^3$ implies that these must exist in $T_uf$. However, we can also show it directly: 
%	\vcomment{at this point if you can prove $\left\{ \frac{\partial \nu}{\partial u_1}, \frac{\partial \nu}{\partial u_2}\right\}$ are linearly independent then do so, or just save for later, in which case don't use derivates at all}
	To show that $\left\{\frac{\partial \nu}{\partial u_1} , \frac{\partial \nu}{\partial u_2}\right\} \in T_u f$,
	first note that at any particular $u \in U$,
	$\inner{\nu}{\nu} = 1 \implies \frac{\partial}{\partial u_i} \inner{\nu}{\nu} = 0$,
	and so by chain rule $2\inner{\frac{\partial \nu}{\partial u_i}}{\nu} = 0
	\implies \frac{\partial \nu}{\partial u_i} \perp \nu $.
	Since $ \nu \perp \spn\left\{\frac{\partial f}{\partial u_i}\right\} $ as well (since $\nu$ its outer product), in  $\R^3$, this implies
	$\spn\left\{\frac{\partial \nu}{\partial u_i}\right\} = % not \parallel
	\spn\left\{\frac{\partial f}{\partial u_i}\right\}$.
	
	Thus, we have $\textrm{span}\left\{ \frac{\partial \nu}{\partial u_1}, \frac{\partial \nu}{\partial u_2}\right\} \subset T_u f$ as well and we can also use it as a basis.
  The Gauss map is intrinsic to the surface and irrespective of any particular curve through it.
	
	We are finally ready to prove \cref{thm:meusnier}, the Theorem of Meusnier.
	
	\begin{proof}
	Let $X\in T_u f$ be given and consider some curve where
  %$\frac{\partial c}{\partial t}(u) = X$ where $X \in T_u f$.
  $c'(p) = X$.
  We wish to decompose the curve's acceleration along the  orthogonal vectors $X$ and
	the Gauss map $\nu = \nu(u_1, u_2) =
		\frac{\frac{\partial f}{\partial u_1} \times \frac{\partial f}{\partial u_2}}
		{\vnorm{\frac{\partial f}{\partial u_1} \times \frac{\partial f}{\partial u_2}}}$ as in \cref{def:gauss-map}.
		Note that $X$ and $\nu$ are indeed orthogonal,
		as $ X \in \spn\left\{\frac{\partial f}{\partial u_i}\right\} = T_u f$, and
		$\nu \perp T_u f$).
	 We then have (at this fixed point $u=\theta(p)$)
		
		\begin{equation} \label{eq:meusnier_acceleration}
			c'' = \inner{c''}{X} X + \inner{c''}{\nu}\nu
			\end{equation}. 
	
	Because $c$ is a regular curve, we either have $c''=0$,
	or $c' \perp c''$, since $\vnorm{c'} = 1$ implies
	$0 = \frac{d}{dt} \inner{c'}{c'} = 2 \inner{c''}{c'} $. Thus
	
		\[ \inner{c''}{X} = \inner{c''}{c'} = 0 \]

	
	 and we can rewrite the second coefficient of \cref{eq:meusnier_acceleration} using the chain rule: % prove chain rule?
	\begin{gather}
  \begin{aligned}
		\inner{c''}{\nu} &=
		\frac{\partial}{\partial t}\left[ \inner{c'}{\nu} \right]
			- \inner{c'}{\frac{\partial \nu}{\partial t}} \\
			&= \frac{\partial}{\partial t}\left[ \inner{X}{\nu} \right]
			- \inner{c'}{\frac{\partial \nu}{\partial t}} \\
			&= 0 - \inner{X}{\frac{\partial \nu}{\partial t}}
			%&= \inner{X}{\wein X} = \mathbf{II}(X,X)
      \end{aligned}
			\end{gather}
	
	Thus, we can express the curvature at this point on our selected curve as
	\begin{align}
	%\vnorm{c''} = \mathbf{II}(X,X) \vnorm{\nu} = \mathbf{II}(X,X)
	\vnorm{c''} = \vnorm{\inner{c''}{X} X + \inner{c''}{\nu}\nu}
	&= \vnorm{0 + \inner{c''}{\nu}\nu}  \nonumber \\
	&= \abs{- \inner{X}{\frac{\partial \nu}{\partial t}}}\vnorm{\nu}  \nonumber \\
	&= \abs{- \inner{X}{\frac{\partial \nu}{\partial t}}}  \nonumber \\
	&=  \abs{\inner{X}{-\frac{\partial \nu}{\partial t}}} \label{eq:expand_accel_norm}
	\end{align}
	
  We may compute the quantity $-\frac{\partial \nu}{\partial t}$ that appears in
	\cref{eq:expand_accel_norm} via chain rule:
	\begin{align} \label{gaussmap_timederivative}
	-\frac{d \nu}{dt} &= -\frac{d}{dt}\big[\nu(u_1, u_2)\big] \nonumber \\
	&= -\frac{d}{dt}\big[\nu(\theta_1(t), \theta_2(t))\big] \nonumber \\
	&= \theta_1'(t)\left( - \frac{\partial \nu}{\partial u_1} \right) + 
	\theta_2'(t)\left( - \frac{\partial \nu}{\partial u_2} \right)
	\end{align}
	
	
	Since $\textrm{span}\left\{ - \frac{\partial \nu}{\partial u_i}\right\}_{i=1,2}$ as a subset of $T_u f$,  we can identify a linear transformation which maps the basis
	$\left\{ \frac{\partial f}{\partial u_i}\right\}_{i=1,2}$ to this basis.
  
  We call this map which the Weingarten map $\wein$.
	
	\begin{defn}[The Weingarten Map] \label{def:wein-map}
		\[
		\wein: T_u f \to T_u f 
		\quad \textrm{given by the composition} \quad
		\wein = D\nu \circ (Df)^{-1}.
		%\wein(\frac{\partial f}{\partial u_i})
		%	= - \frac{\partial \nu}{\partial u_i}
		%\quad i=1,2		
		\]
	\end{defn}
	That is, $\wein(\frac{\partial f}{\partial u_i}) = - \frac{\partial \nu}{\partial u_i} $ for $i=1,2$.
  %where the negative sign comes about from blind adherence to \cref{eq:gaussmap_dt_expanded} and \cref{eq:expand_accel_norm}.
  This allows us to rewrite the time derivative of the Gauss map \cref{gaussmap_timederivative} as
		\begin{align}
		-\frac{d \nu}{dt} &= 
		\theta_1'(t)\left( - \frac{\partial \nu}{\partial u_1} \right) + 
		\theta_2'(t)\left( - \frac{\partial \nu}{\partial u_2} \right) \nonumber\\
		&= \theta_1'(t)\left( \wein\left(\frac{\partial f}{\partial u_1}\right) \right) + 
		\theta_2'(t)\left( \wein\left(\frac{\partial f}{\partial u_2}\right) \right)\nonumber \\
		&= \wein\left[\theta_1'(t)\left( \frac{\partial f}{\partial u_1}\right)  + 
		\theta_2'(t)\left( \frac{\partial f}{\partial u_2} \right)\right] \nonumber\\
		&= \wein\left(\frac{d}{dt}\left[ f\left(\theta(t)\right)\right] \right)
		= \wein\left( \frac{d}{dt} \left[ c(t) \right] \right) = \wein\left(X\right)
		\end{align} 
		With this, we can re-express the curvature of our curve from
		\cref{eq:expand_accel_norm} as the much simplified
		
		\begin{equation} \label{eq:expand_accel_norm_final}
		\vnorm{c''} = \abs{\inner{X}{-\frac{\partial \nu}{\partial t}}}
                = \abs{\inner{X}{\wein(X)}}
		\end{equation}
		
	
	The linear transformation $\wein$ from \cref{def:wein-map}, and thereby
	the computation of curvature given in \cref{eq:expand_accel_norm_final},
	depends only on the point $u$ and the selected direction $X$, not on the particular curve $c$ at all.
	\end{proof}
	To recap, given a point $u$ on the surface and an arbitrary vector $X$ in the tangent plane, we can calculate the curvature of any surface curve with velocity $X$ there. In fact, we refer to this intrinsic quantity as the normal curvature of the surface.
	
	\begin{defn} \label{def:normal-curvature}
		The normal curvature of a surface, denoted $\kappa_\nu$ at point $u$ in the direction $X$ is given by
		\[\kappa_\nu :=  \inner{X}{\wein(X)} \]
	\end{defn}
	In fact, \cref{thm:meusnier} shows that the normal curvature is an intrinsic property of the surface--it depends only on the
	surface at a point, and no reference to any particular curve on the surface is necessary or implied.
%	In other contexts (not necessary here), this quantity is referred to as the \textit{second fundamental form} at the point $u \in U$; that is, $ \mathbf{II}(X,X) := \inner{X}{\wein(X)}$.
	
	
	The map $\wein$ introduced in the proof above is known as the Weingarten map
	and is implicitly defined at each $u \in U$. 
	We wish to make its existence rigorous as well as find a matrix representation for it, using the standard motivation that $\wein(\frac{\partial f}{\partial u_i}) = - \frac{\partial \nu}{\partial u_i}$.
	
	
	That is, we may trace any $X \in T_u f$ which has been expanded in terms of the basis 
	$\left\{\frac{\partial f}{\partial u_1} , \frac{\partial f}{\partial u_2}\right\}$
	and map it to the basis $\left\{-\frac{\partial \nu}{\partial u_1} , -\frac{\partial \nu}{\partial u_2}\right\}$. 
	
	The Weingarten map can be formally shown to be well-defined, invariant under coordinate transformation in the general case (that is, for surfaces $f$ that are not graphs). We refer to \cite{Kuhnel-DiffGeo} for the general proof. Our present situation is much less delicate, as we're only concerned for cases when $f$ is a graph. In this case, the linear transformation may be simply constructed, and we proceed by simply calculating its matrix representation.	
	\begin{lemma}
		The Weingarten map as in \cref{def:wein-map} is well-defined for graphs.
	\end{lemma}
	To find a matrix representation for $\wein$ (which we will denote $\weinmat \in \R^{2\times2}$), we simply wish to find a linear transformation
	such that
	$\weinmat \left.\frac{\partial f}{\partial u_i}\right|_{T_u f}
		= - \left.\frac{\partial \nu}{\partial u_i}\right|_{T_u f} \quad \textrm{for} \; i=1,2$
			where $- \left.X\right|_{T_u f}$ denotes that $X \in T_u f$ is being represented in 
	local coordinates for $T_u f$ (Strictly speaking, of course $T_u f \subset \R^3$ and thus
	$\frac{\partial f}{\partial u_i} \in \R^3.$ Thus when we say $ \left.\frac{\partial f}{\partial u_i}\right|_{T_u f}$ we are referring to this 3-vector expanded with respect to the two-dimensional basis for $T_u f$). In matrix form, we describe this situation as
	
	\begin{align}
	\Bigg[ \;\weinmat\; \Bigg]
	\left[ \mcol{\left.\frac{\partial f}{\partial u_1}\right|_{T_u f}} \;
			\mcol{\left.\frac{\partial f}{\partial u_2}\right|_{T_u f}} \right]
			&= \left[ \weinmat \mcol{\left.\frac{\partial f}{\partial u_1}\right|_{T_u f}} \;
			\weinmat \mcol{\left.\frac{\partial f}{\partial u_2}\right|_{T_u f}} \right] \\
			&= \left[ \mcol{ \left.-\frac{\partial \nu}{\partial u_1}\right|_{T_u f}} \;
			\mcol{\left.-\frac{\partial \nu}{\partial u_2}\right|_{T_u f}} \right]
			\end{align}
			Now, representing each vector in  $T_u f$ with respect to the basis $\left\{ \frac{\partial f}{\partial u_i}\right\}$, we have
  \begin{align}
      	\Bigg[ \;\weinmat\; \Bigg]
      \left[ \mcol{\left.\frac{\partial f}{\partial u_1}\right|_{T_u f}} \;
      \mcol{\left.\frac{\partial f}{\partial u_2}\right|_{T_u f}} \right]
      = 
			\Bigg[ \;\weinmat\; \Bigg]
			\begin{bmatrix} - \frac{\partial f}{\partial u_1} - \\
				- \frac{\partial f}{\partial u_2} -
							\end{bmatrix}
			\left[ \mcol{\frac{\partial f}{\partial u_1}} \;
				\mcol{\frac{\partial f}{\partial u_2}}\right]
				= \begin{bmatrix} - \frac{\partial f}{\partial u_1} - \\
				- \frac{\partial f}{\partial u_2} -
				\end{bmatrix}
				\left[ \mcol{-\frac{\partial \nu}{\partial u_1}} \;
				\mcol{-\frac{\partial \nu}{\partial u_2}}\right] \\
	\end{align}
		 
	We can simplify this greatly by defining 
	\begin{equation}\label{fundamentalformcoefficients}
	g_{ij} := \inner{\frac{\partial f}{\partial u_i}}{\frac{\partial f}{\partial u_j}}
	\quad \textrm{and}\quad
	h_{ij} := \inner{\frac{\partial f}{\partial u_i}}{-\frac{\partial \nu}{\partial u_j}}
	\end{equation}
	
	so that
	\begin{equation}
	\Bigg[ \;\weinmat\; \Bigg]
	\begin{bmatrix} g_{11} & g_{12} \\ g_{21} & g_{22} \end{bmatrix}
	= \begin{bmatrix} h_{11} & h_{12} \\ h_{21} & h_{22} \end{bmatrix}
	\end{equation}
	
	Then we rearrange to solve for $\weinmat$ as
		\begin{equation} \label{weinmatconstruction}
		\weinmat
		\;=\; \begin{bmatrix} h_{11} & h_{12} \\ h_{21} & h_{22} \end{bmatrix}
		\left.\begin{bmatrix} g_{11} & g_{12} \\ g_{21} & g_{22} \end{bmatrix}\right.^{-1}
		\end{equation}
	
	where $\left[ g_{ij} \right]$ is clearly invertible, as the set
	$\left\{\frac{\partial f}{\partial u_j}\right\}$ is linearly independent.
	
	It should be noted that this matrix representation is accurate not only for the surface of a graph, but for any \textit{generalized} surface
	$f: U \to \R^3 $ with $u \mapsto (x(u), y(u), z(u))$ as well. We shall later show that this calculation simplifies (somewhat) in the case that our surface is a graph.
	
	Our final goal is to characterize such normal curvatures.
	Namely, we wish to establish a method of determining in which directions an extremal normal curvature occurs.
	
	
	\subsection{Principal Curvatures and Principal Directions}
	To do so, we shall consider the relationship between the direction $X$ and the normal curvature $\kappa_\nu$ \cref{def:normal-curvature} in that direction at some specified $u$.


	
	First, we need the following lemma:
    \begin{lemma}
        If $A\in R^{n\times n}$ is a symmetric real matrix, $v \in R^n$
        and given the dot product $\inner{\cdot}{\cdot}$,
        we have $\nabla_{v} \inner{v}{Av} = 2Av$.
        In particular, when $A = I$ the identity matrix, we have
        $ \nabla_{v} \inner{v}{v} = 2v$.
    \end{lemma}
    
    \begin{proof}
        The result is uninterestingly obtained by tracking
        each component of
        $\nabla_{v} \inner{v}{Av}$:
        
        \begin{align}
        {\Big(\nabla_{v} \inner{v}{Av}\Big)}_{i} =
                \frac{\partial}{\partial v_i} \Big[
                \inner{v}{Av}\Big]
                &=  \frac{\partial}{\partial v_i} \left[
                    \sum_{j=1}^{n} v_j {(Av)}_j\right] \\
                &=	\frac{\partial}{\partial v_i} \left[
                \sum_{j=1}^{n} v_j \sum_{k=1}^n a_{jk} v_k \right] \\
                &= \frac{\partial}{\partial v_i} \left[
                a_{ii} v_i^2 + v_i \sum_{k\ne i} a_{ik} v_k
                 + v_i \sum_{j\ne i} a_{ji} v_j
                 + \sum_{j\ne i} \sum_{k \ne i} v_j a_{jk} v_k \right] \\
    &=  2 a_{ii} v_i + \sum_{k\ne i} a_{ik} v_k
    + \sum_{j\ne i} a_{ji} v_j + 0 \\
    &= 2 a_{ii} v_i + 2 \sum_{k\ne i} a_{ik} v_k
     = 2 \sum_{k=1}^n a_{ik}v_{k} = 2 {\big(Av\big)}_i \\
     &\quad \implies \nabla_{v} \inner{v}{Av} = 2Av.
        \end{align}
    \end{proof}
    
    We are now ready for the major result of this section, which ties the Weingarten map to the
    notion of normal curvatures.
    
    \begin{theorem}[Theorem of Olinde Rodrigues]
       	Fixing a point $u \in U$, a direction $X \in T_u f $ minimizes the normal curvature $\kappa_\nu = \inner{\wein X}{X}$ subject to $\inner{X}{X}= 1$
       	iff $X$ is a (normalized) eigenvector of the Weingarten map $\wein$.
       	\end{theorem}
    \begin{proof}
       	%	Recall first the definition of the first two fundamental forms,
       		%$\textbf{II}(X,X) = \inner{\wein{X}}{X}$
       	%	and $\textbf{I}(X,X) = \inner{X}{X}$.
       		
       		In the following, we will assume that $X \in T_u f$ is expanded,
       		in local coordinates, i.e. along  a two dimensional basis
       		(such as $\left\{ \frac{\partial f}{\partial u_i}\right\}_{i=1,2}$
       		) and thus can refer to $\wein$ freely as the $2\times2$ matrix $\weinmat$.
       		Using the method of Lagrange multipliers, we define the Lagrangian:
       		\begin{equation}
       		\mathscr{L}(X; \lambda) :=
        	\inner{\weinmat X}{X} - \lambda\Big(\inner{X}{X} - 1\Big) 
     \end{equation}
        	
        	Extremal values occur when
        	$\nabla_{X,\lambda} \mathscr{L}(X;\lambda) = 0$,
        	which results in the two equations
        	
     \begin{equation} \label{lagrange_requirements}
     \left\{ \begin{aligned}
      \nabla_X \inner{\weinmat X}{X} - \lambda \nabla_X \left( \inner{X}{X} - 1 \right) = 0 \\
      \inner{X}{X} - 1 = 0
     \end{aligned} \right.
     \end{equation}
     
     The second requirement is simply the constraint that $X$ is normalized.
     Using the previous lemma, we can simplify the first result as follows:
     
     \begin{gather}
     \nabla_X \inner{\weinmat X}{X} - \lambda \nabla_X \left( \inner{X}{X} - 1 \right) = 0 
     \nonumber \\
     2\weinmat X - \lambda \left(2 X \right) = 0  \nonumber \\
     \implies \quad \weinmat X - \lambda X = 0 \nonumber \\
     \implies \quad \weinmat X = \lambda X
     \end{gather}
     which implies that $X$ is an eigenvector of  $\weinmat$ with corresponding eigenvalue $\lambda$ ($X\ne 0 $ from the second equation of \cref{lagrange_requirements}).
     Thus the two hypotheses are exactly equivalent when $X$ is normalized. It is also worth remarking that the corresponding eigenvalue $\lambda$ is the Lagrangian multiplier itself.
       	\end{proof}
       	
       	Thus, to find the directions of greatest and least curvature of a surface at a point $u \in U$, we simply must calculate the Weingarten map and its eigenvectors. We refer to these directions as follows.
       	
       	\begin{defn}[Principal Curvatures and Principal Directions]
       		The extremal values of normal curvature of a surface at a point  $u\in U$
       		are referred to as \textbf{principal curvatures}. The corresponding directions at which normal curvature attains an extremal value are referred to as \textbf{principal directions}.
       	\end{defn}
       	
       	Our final goal is to explicitly determine a (hopefully simplified) version of the Weingarten map in the case of a graph $f(u_1,u_2) = (u_1,u_2, h(u_1,u_2))$ and calculate
       	the principal directions and curvatures in a simple example.
       	        		
       	\begin{theorem}[Weingarten Map of a Graph]
        	Given the graph $f: U \to \R^3$ where $(x,y) \mapsto (x, y, h(x,y))$, the matrix
        	representation of its Weingarten map is given by
        	\begin{equation} \label{weinmatexactgraph}
        	\weinmat = \Hess(h) \tilde{G} \;,\quad \mathrm{where} \quad
	        	\tilde{G} := \frac{1}{\left({1+h_x^2 + h_y^2}\right)^{3/2}}
	        	\begin{bmatrix}
		        	1 + h_y^2 & -h_x h_y \\
		        	-h_x h_y & 1 + h_x^2 \\
	        	\end{bmatrix} 
        	\end{equation}
        	
        	In particular, given a point $u = (x,y) \in U \subset \R^2$ where $h_x \approx h_y \approx 0$, we
        	have $\tilde{G} \approx \mathrm{Id}$, and thus $\weinmat \approx \Hess(h)$.
        	
       	\end{theorem}
       	\begin{proof}
       		We begin from \cref{weinmatconstruction}.
       		First, consider each component from \cref{fundamentalformcoefficients}
       		and rewrite via chain rule:
      		 \[ h_{ij} = \inner{\frac{\partial f}{\partial u_i}}{-\frac{\partial \nu}{\partial u_j}}
       		= \inner{\frac{\partial^2 f}{\partial u_i \partial u_j}}{\nu} \]
       		
       		Now, given our particular surface $f$, we can calculate each of these components directly. We have:
       		\begin{equation}
       		\begin{gathered}
       		f_{x} = (1, 0, h_x) , \quad
       		f_{y} = (0, 1, h_y)  \\
       		f_{xx} = (0, 0, h_{xx}) , \quad
       		f_{xy} = (0, 0, h_{xy}) = f_{yx} , \quad
       		f_{yy} = (0, 0, h_{yy})
       		\end{gathered}
       		\end{equation}
       		
       		and we have the unit normal vector (Gauss map)
       		\begin{align}
       		\nu(u_1, u_2) &=
       		\frac{\frac{\partial f}{\partial x} \times \frac{\partial f}{\partial y}}
       		{\vnorm{\frac{\partial f}{\partial x} \times \frac{\partial f}{\partial y}}} \\
       		&= \frac{(1, 0, h_x) \times (0, 1, h_y)}{\vnorm{(1, 0, h_x) \times (0, 1, h_y)}} \\
       		&= \frac{\left( -h_x , -h_y , 1 \right)}{\sqrt{h_x^2 + h_y^2 + 1}}
	 \end{align}
	 We then calculate each $h_{ij}$ as
	 \begin{equation}
	 \begin{gathered}
	 h_{11} = \inner{\frac{\partial^2 f}{\partial x^2}}{\nu} = 
		 \frac{h_{xx}}{\sqrt{1+h_x^2 + h_y^2}} \\
	  h_{12} = \inner{\frac{\partial^2 f}{\partial x \partial y}}{\nu} = 
	  \frac{h_{xy}}{\sqrt{1+h_x^2 + h_y^2}} = h_{21} \\
	  h_{22} = \inner{\frac{\partial^2 f}{\partial y^2}}{\nu} = 
	  \frac{h_{yy}}{\sqrt{1+h_x^2 + h_y^2}} \\
	 \end{gathered}
	 \end{equation}
	 
	 and thus the first matrix in \cref{weinmatconstruction} is given by
	 
	 \begin{equation} \label{hij_exactgraph}
	 [h_{ij}] = \frac{1}{\sqrt{1+h_x^2 + h_y^2}} \,  \Hess (h)
	 \end{equation}
	 
	 To calcuate the second, we use
	 
	 
	 
	 \begin{equation}
	 \begin{gathered}
	 g_{ij} = \inner{\frac{\partial f}{\partial u_i}}{\frac{\partial f}{\partial u_j}} \\
	 g_{11} = \inner{f_x}{f_x} = 1 + h_x^2 \\
	 g_{12} = \inner{f_x}{f_y} = h_x h_y = g_{21} \\
	 g_{22} = \inner{f_y}{f_y} = 1 + h_y^2
	 \end{gathered}
	 \end{equation}
	 
	 and thus
		\begin{equation} \label{gij_exactgraph}		 
		[g_{ij}]^{-1} = \begin{bmatrix} 1 + h_x^2 & h_x h_y \\
					h_x h_y & 1 + h_y^2 \end{bmatrix}^{-1}
					\;=\;	\frac{1}{1+h_x^2 +h_y^2}
					\begin{bmatrix} 1 + h_y^2 & -h_x h_y \\
					-	h_x h_y & 1 + h_x^2 \end{bmatrix}
		\end{equation}
       	
       	Combining $[h_{ij}]$ and $[g_{ij}]^{-1}$ from \cref{gij_exactgraph} and \cref{hij_exactgraph}
       	we arrive at our result, \cref{weinmatexactgraph}.
       	\end{proof}
       	
  We stress that this map $\wein$ is defined for each point $u \in U$.
  In the particular case that $u \in  U$ is a critical point, where $\nabla h(u) = (h_x(u), h_y(u)) = 0$, then it is clear from the previous theorem that $\weinmat$ is exactly the Hessian matrix $\Hess(h)$. Of course this implies that $\weinmat$ and $\Hess{(h)}$ have the same eigenvalues and eigenvectors at any such point.

%FIX_THIS % % % % % % % % % % % % % % % % % % % % % % % %
%But this observation is more broadly useful than for analyzing critical points alone. If $\tilde{G}$ above is close to identity, then the eigenvalues and eigenvectors of $\weinmat$ will be similarly close to the eigenvalues of the Hessian. We can rewrite $\tilde{G}$ from \cref{weinmatexactgraph} as identity plus a small matrix:
%\begin{equation}
%\tilde{G} = \frac{1}{1+h_x^2 +h_y^2}\left(I + [\delta]\right), \quad
%	[\delta] 	:= \begin{bmatrix} h_y^2 & -h_x h_y \\
%					  -h_x h_y & h_x^2 \end{bmatrix}
%\end{equation}
%
%We can then rewrite \cref{weinmatexactgraph} as 
%\begin{equation}
%\weinmat = \frac{1}{\left(1+h_x^2 + h_y^2\right)^{3/2}}\left( \Hess(h)
%+  \Hess(h) [\delta]\right)
%\end{equation}
%
%We can see that as $h_x, h_y$ are close to zero, $[\delta]$ will be very close to the zero matrix, and the constant $\left(1+h_x^2 + h_y^2\right)^{-3/2}$ will be very close to 1 as well, so we should not expect the addition of a "close to 0" matrix to have much effect on the eigenvectors or eigenvalues. This intuition is confirmed by a result from Wilkinson \cite{wilkinson-eigenvalue}, which we state without rigorous proof.
%
%% see wilkinson pg 66
%\begin{theorem}
%	If $A$, $B$ are matrices such that $\abs{A_{ij}} < 1 , \abs{B_{ij}} < 1$ (a condition that can be ignored with scaling) and $\lambda$ is a simple eigenvalue of $A$, then given $\epsilon > 0$, there exists a simple eigenvalue $\tilde{\lambda}$ of the matrix $A + \epsilon B$ with $\abs{\lambda - \tilde{\lambda}} = \bigO(\epsilon)$. Similarly, if $v$ is an eigenvector of A, then $\tilde{v}$ is an eigenvector of $A + \epsilon B$ with
%	$\abs{v - \tilde{v}} = \bigO(\epsilon)$.
%\end{theorem}
%
%The proof ultimately relies on a general result of analysis, that the zeros of a polynomial are continuous with respect to its coefficients. In this case, the polynomial in question is the characteristic polynomial
%$p(\lambda) = \det(\lambda I - A - \epsilon B)$, whose coefficients will scale with $\epsilon$. Thus $\weinmat \approx \Hess(h)$ for any point where the gradient $\nabla h \approx 0$. We shall see that we're only concerned with regions where $h_x, h_y$ is small anyway, and we do not expect much response anyway when the gradient is large. 
%
%We can bound the perturbation of eigenvalues from $\weinmat$ to $\Hess(h)$ by another result
%which we state without proof \cite{matrix-analysis-horn}.
%
%\begin{theorem}
%	Let A be a $n\times n$ normal matrix with eigenvalues $\lambda_1, \dots, \lambda_n$ and $E$ an $n \times n$ arbitrary matrix. If $\hat{\lambda} $ in an eigenvalue of $A + E$, then there is an eigenvalue $\lambda_i$ of A for which
%		$\abs{\hat{\lambda} - \lambda_i} \leq \vnorm{E}_2$
%\end{theorem}
%
%In the event that we do wish to rigorously compute the Weingarten map--that is, without concern for the magnitude of the gradient--we refer to \cite{jiao2008consistent} and survey papers mentioned therein.

  
  To make the Weingarten map and its relationship to the Hessian more explicit, we will calculate the Weingarten map for a relatively simple graph.
  
  \subsection{The Weingarten map and Principal Curvatures of a Cylindrical Ridge} \label{sec:calculate-weinmap-of-a-ridge}
  
  Let $f$ be the graph given by 
  \begin{equation}
   f: \R^2 \to \R^3 \;\textrm{by}\; f(x,y) = (x,y,h(x,y)), \;\textrm{with}\;
   h(x,y) = \begin{cases}
    \sqrt{r^2 - x^2} & -r < x < r \\
    0 & \textrm{else}
    \end{cases}
  \end{equation} 
  \begin{figure}[h!]
  \includegraphics[width=\linewidth]{circular_trough_with_axes_vectors}
  \caption{The graph of a cylindrical ridge of radius $r$}
  \label{fig:ridge-graph}
  \end{figure}
  
  
  The graph $f$ is depicted in \cref{fig:ridge-graph}. We calculate the necessary partial derivatives of $h(x,y)$ and apply them to \cref{weinmatexactgraph}.
  
%  \begin{gather}
%  \frac{\partial f}{\partial x} = \left(1, 0, \frac{-x}{\sqrt{r^2 - x^2}}\right)
%  \quad , \quad
%  \frac{\partial^2 f}{\partial x^2} = \left(0, 0, \frac{-r^2}{\left(\sqrt{r^2 - x^2}\right)^3}\right) \\
%  \frac{\partial f}{\partial y} = \left(0, 1, 0\right)
%  \quad , \quad
%  \frac{\partial^2 f}{\partial y^2} = \frac{\partial^2 f}{\partial x \partial y} = 0
%  \end{gather}
%  The gauss map is given by
%  \begin{gather}
%  \nu(x,y) = \frac{\frac{\partial f}{\partial x} \times \frac{\partial f}{\partial y}}
%  {\vnorm{\frac{\partial f}{\partial x} \times \frac{\partial f}{\partial y}}}
%  = \left( \frac{x}{r} ,\; 0 ,\; \frac{\sqrt{r^2 - x^2}}{r} \right) \\
%\implies
%\frac{\partial \nu}{\partial x}
% = \left(\frac{1}{r} \;,\; 0\;,\; \frac{-x}{r\sqrt{r^2-x^2}}\right)
% \quad , \quad \frac{\partial \nu}{\partial y} = \left(0,0,0\right).
%  \end{gather}
%  
%  We then calculate matrix elements of the Weingarten map's construction as given in
%  \cref{hij_exactgraph} and \cref{gij_exactgraph} :
%  \begin{align}
%  [h_{ij}] = \frac{1}{\sqrt{1+h_x^2 + h_y^2}} \,  \Hess (h)
%     &= \frac{1}{\sqrt{1+\left(\frac{x^2}{r^2-x^2}\right)}}
%     \begin{bmatrix}
%     \frac{-r^2}{\sqrt{r^2 - x^2}^3} & 0 \\
%    0 & 0
%     \end{bmatrix} 
%     = \begin{bmatrix}
%     \frac{-r}{r^2 - x^2} & 0 \\
%     0 & 0
%     \end{bmatrix} \
%     [g_{ij}]^{-1} &= \begin{bmatrix} \frac{r^2 - x^2}{r^2} & 0 \ 0 & 1 \end{bmatrix} \\
%     \implies \weinmat = [h_{ij}]	[g_{ij}]^{-1} &=
%     \begin{bmatrix}
%     \frac{-r}{r^2 - x^2} & 0 \\
%     0 & 0
%     \end{bmatrix} \begin{bmatrix} \frac{r^2 - x^2}{r^2} & 0 \ 0 & 1 \end{bmatrix} \\
%     &= \begin{bmatrix} - \frac{1}{r} & 0 \\ 0 & 0 	\end{bmatrix}	       	
%  \end{align}
\begin{gather*}
h_x = \frac{-x}{\sqrt{r^2 - x^2}}
\quad , \quad
h_{xx} = \frac{-r^2}{\left(r^2 - x^2\right)^{3/2}}\\
h_y = 0
\; , \quad
h_{yy} = 0 \;,\quad h_{xy} = 0
\end{gather*}

    \begin{align*}
  \weinmat &= \Hess(h)\tilde{G}
  = \frac{1}{\left(1+\frac{x^2}{r^2-x^2}\right)^{3/2}}
  \begin{bmatrix} \frac{r^2}{\left(r^2 - x^2\right)^{3/2}} & 0 \\
  0 & 0 \end{bmatrix} \\
  &= \frac{1}{\left(\frac{r^2}{r^2-x^2}\right)^{3/2}}
  \begin{bmatrix} \frac{r^2}{\left(r^2 - x^2\right)^{3/2}} & 0 \\
  0 & 0 \end{bmatrix}
%  = \frac{\left(r^2-x^2\right)^{3/2}}{r^3}
%  \begin{bmatrix} \frac{-r^2}{\left(r^2-x^2\right)^{3/2}} & 0 \\ 0 & 0 \end{bmatrix}
  = \cdots
  = \begin{bmatrix} \frac{-1}{r} & 0 \\ 0 & 0 \end{bmatrix}
  \end{align*}
  
The matrix of the Weingarten map and its eigenvalues (in local coordinates) and vectors are then
\begin{equation} \label{eq:trough-example-eigens}
  \weinmat = \begin{bmatrix} - \frac{1}{r} & 0 \\ 0 & 0 	\end{bmatrix} \quad
  u_2 = (1,0) \;,\; u_1 = (0,1) \quad
  \lambda_2 = \frac{-1}{r} \;,\; \lambda_1 = 0
  \end{equation}

The Theorem of Olinde Rodriguez then suggests that $u_2$ points in the direction of maximum curvature of the surface, $-\frac{1}{r}$, which is predictably in the direction directly perpendicular to the trough, whereas the direction of least curvature is along the trough and the curvature there is $0$. The Theorem of Meusnier, \cref{thm:meusnier}, suggests that the normal curvature $\kappa_2 = -\frac{1}{r}$ is reasonable--any curve on the trough perpendicular to the ridge should have the curvature of a circle, and the negative simply indicates that we are on the ``outside'' of the surface. Finally, we note that at the ridge of the trough is exactly where $\nabla f = 0$, and the Weingarten map is exactly the Hessian matrix there.

For arbitrary points $-r < x < r$, the Hessian takes the general form

\begin{equation}
\Hess(h) =
  \begin{bmatrix} \frac{-r^2}{\left(r^2 - x^2\right)^{3/2}} & 0 \\
                  0 & 0 \end{bmatrix}
\end{equation}

Its two eigenvectors are the same as in \cref{eq:trough-example-eigens},and its eigenvalues are $\lambda_2 = \frac{-r^2}{\left(r^2 - x^2\right)^{3/2}}$ and $\lambda_1 = 0$ and \cref{fig:ridge-graph}. Note that the leading eigenvalue is now a function of $x$.
% unfortunately, it's actually taking a maximal value as x is farther away from $x=0$ which is kind of unintuitive but i won't mention it here

%  \section{Orthogonality of Principal Directions and Approximated Principal Directions}
%    
%  \vcleanup{the eigenvectors of the Weingarten map are orthonormal. the eigenvectors of the hessian are also orthonormalizable. make the distinction!}
%  Our final point is to consider the relationship between the principal directions of a surface at a point. Viewing the surface in $\R^3$, we define the Hessian $\Hess(x,y)$ of the surface $L$
%  at a point $(x,y)$ on the surface as the matrix of its second partial derivatives:
%  \begin{equation}
%  \Hess(x,y) = \begin{bmatrix}
%  L_{xx}(x,y) & L_{xy}(x,y) \\
%  L_{yx}(x,y) & L_{yy}(x,y)
%  \end{bmatrix}
%  \end{equation}
%  
%  At any point $(x,y)$ we denote the two eigenpairs of the Hessian matrix of $h$ as
%  \begin{equation}
%  \Hess(x,y) u_i = \kappa_i u_i \; , \quad i = 1,2
%  \end{equation}
%
%  where $\kappa_i$ and $u_i$ are known as the
%  \textit{approximated principal curvatures} and \textit{approximated principal directions} of $L(x,y)$, respectively, and we label such that $|\kappa_2| \ge |\kappa_1|$. Notably, $\Hess(x,y)$ is a real, symmetric matrix (since  $L_{xy} = L_{yx}$ and $L$ is a real function) and thus its eigenvalues are real and its eigenvectors are orthonormal to each other, as given by following basic result from linear algebra
%  \cite{burden-faires}.
%  
%  % from burden and faires corollary 9.17
%  \begin{lemma}
%     	Let $A$ be a real, symmetric matrix. The eigenvalues of $A$ are real and its eigenvectors are orthonormal to each other.
%  \end{lemma}
%  
%  \begin{proof}
%     	Let $x\ne 0$ so that $Ax = \lambda x$. Then 
%     	\begin{align*}
%     	\vnorm{Ax}_2^2 = \inner{Ax}{Ax}  &= (Ax)^{*} Ax \\
%     	&= x^{*}A^{*}Ax = x^{*}A^T A x = x*A A x \\
%     	&= x^{*} A \lambda x = \lambda x^{*} A x \\
%     	&= \lambda x^{*} \lambda x = \lambda^2 x^{*} x = \lambda ^2 \vnorm{x}_2^2
%     	\end{align*}
%     	Upon rearrangement, we have
%     	$\lambda^2 = \frac{\vnorm{Ax}_2^2}{\vnorm{x}_2^2} \ge 0 \implies \lambda $ is real.
%     	
%     	To prove that a set of orthonormalizable eigenvectors exists,
%     	let $A$ be real, symmetric as above and consider the eigenpairs
%     	$Av_1 = \lambda_1 v_1$, $Av_2 = \lambda_2 v_2$ with $v_1, v_2 \ne 0$.
%
%     	5
%     	In the case that $\lambda_1 \ne \lambda_2$, we have
%     	\begin{align*}
%     	(\lambda_1 - \lambda_2)v_1^T v_2 &= \lambda_1 v_1^T v_2 - \lambda_2 v_1^T v_2 \\
%     	&= (\lambda_1 v_1)^T v_2 - v_1^T (\lambda_2 v_2) \\
%     	&= (Av_1)^T v_2 - v_1^T (Av_2) \\
%     	&= v_1^T A^T v_2 - v_1^T A v_2 \\
%     	&= v_1^T A v_2 - v_1^T A v_2 = 0
%     	\end{align*}
%     	Since $\lambda_1 \ne \lambda_2$, we conclude that $v_1^T v_2 = 0$.
%     	
%     	In the case that $\lambda_1 = \lambda_2 =: \lambda$, we can define
%     	(as in Gram-Schmidt orthogonalization) $u = v_2 - \frac{v_1^Tv_2}{v_1^Tv_1}v_1$.
%     	This is an eigenvector for $\lambda=\lambda_2$, as
%     	\begin{align*}
%     	Au &= A\left(v_2 - \frac{v_1^Tv_2}{v_1^Tv_1} v_1\right) \\
%     	&= A v_2 - \frac{v_1^Tv_2}{v_1^Tv_1} A v_1 \\
%     	&= \lambda v_2- \frac{v_1^Tv_2}{v_1^Tv_1} \lambda v_1 \\
%     	&= \lambda \left( v_2 - \frac{v_1^Tv_2}{v_1^Tv_1} v_1 \right) = \lambda u
%     	\end{align*}
%     	and is perpendicular to $v_1$, since
%     	\begin{align*}
%     	v_1^T u &= v_1^T\left(v_2 - \frac{v_1^Tv_2}{v_1^Tv_1}v_1\right) \\
%     	&= v_1^T v_2 - \left(\frac{v_1^Tv_2}{v_1^Tv_1}\right) v_1^T v_1 \\
%     	&= v_1^T v_2 - v_1^Tv_2 (1) = 0.
%     	\end{align*}
%  \end{proof}
%  
%  Thus we see that the two principal directions form an orthonormal frame at each point (x,y) within the continuous image $L(x,y)$.
  






    
   
    
   
 
 

\chapter{The Uniscale Frangi Filter} \label{ch:unifrangi}

	We now seek to harness the ideas introduced in \cref{ch:diffgeo} to the task at hand: identifying curvilinear content within images.

The Frangi filter, first described by Alejandro Frangi et al., is a widely used  Hessian-based filter
within image processing \autocite{frangi-paper}. Hessian-based filters make use of the
logical ``proximity'' of the Hessian to notions of curvature of surfaces. Frangi's filter was orignally developed for vascular segmentation in images such as MRIs and it excels in that context \autocite{frangi-paper}.
Several alternatively formulated Hessian-based filters exist \autocite{sato-filter,lorenz-filter,olabarriaga-hessian-comparison}. These filters use information about the principal curvatures, approximated as eigenvalues of the Hessian, at each point in the image
to identify regions of significant curvature within an image.

We point out here that our image is of course a discrete structure and not a continuous surface, which was the object of concern in \cref{ch:diffgeo}.
In \cref{ch:scale-space-theory}, we will develop the notions of scale space, which will help us bridge the gap between continuous surfaces and discrete images. Using that approach, we shall consider the Frangi filter in its standard multiscale setting. For the time being, we continue to consider our images as continuous graphs, as the theory of the Frangi filter in its basic sense does not require the extra machinery of scale space theory.


The procedure for a continuous 2D image is as follows:
Let $\lambda_1, \lambda_2$ be the two eigenvalues of the Hessian of the image at point $(x, y)$,
ordered such that $\abs{\lambda_1} \leq \abs{\lambda_2}$, and define the Frangi vesselness measure for identifying bright curvilinear structures as %at scale $\sigma$ as:

\begin{equation} \label{eq:frangi-vesselness-measure}
\V(x,y) = \begin{cases}
0 & \text{if} \quad \lambda_2 > 0 \\
\exp\left\{-\frac{\Am^2}{2\beta^2}\right\}
\left(1 - \exp\left(-\frac{\Sm^2}{2c^2}\right)\right) & \text{otherwise}
\end{cases} \end{equation}
where
\begin{equation} \label{frangi-def-anisotropy-structureness}
\Am := \abs*{\frac{\lambda_1}{\lambda_2}}
\quad \textrm{and} \quad 
\Sm := \sqrt{\lambda_1^2 + \lambda_2^2}
\end{equation}
and $\beta$ and $c$ are tuning parameters. $\Am$ and $\Sm$ are known as the anisotropy measure and structureness measure, respectively. Similarly, we shall refer to the two factors in \cref{eq:frangi-vesselness-measure} as the anisotropy factor and structureness factor. We will also refer to $\lambda_2$ and $\lambda_1$ as the leading and trailing eigenvalues of the Hessian, respectively. Before we discuss appropriate values for $\beta$ and $c$, we first seek to highlight the significance of the two measures in \cref{frangi-def-anisotropy-structureness}.

\section{Anisotropy Measure} \label{sec:frangi.anisotropy}
The anisotropy (or directionality) measure $\Am$ is simply the ratio of magnitudes of $\lambda_1$ and $\lambda_2$. Since at a ridge point of a tubular structure, we should have $\lambda_1 \approx 0$ and $\abs*{\lambda_2} \gg \abs*{\lambda_1}$,
a very small value of $\Am$ would be present at a ridge of a tubular structure.
In \cref{fig:circular_trough_with_axes_vectors}, this situation is demonstrated. Here, $u_1$ and $u_2$ form the orthogonal set of Hessian eigenvectors with corresponding eigenvalues $\lambda_1$ and $\lambda_2$. At such a ridgelike structure, we could predict the largest change in curvature to be straight down the ridge (in the direction of $u_2$), and the direction of least curvature to be directly along the ridge (in the direction  of $u_1$). The leading eigenvalue $\lambda_2$ should be large and negative, and $\lambda_1$ should be approximately zero.

\begin{figure} \centering
  \includegraphics[width=0.8\linewidth]{circular_trough_with_axes_vectors}
  \caption{The principal eigenvectors at a ridge-like structure} 
  \label{fig:circular_trough_with_axes_vectors}
\end{figure}


Of course, if the the ridge is perfectly circular along its cross section, as in \cref{sec:calculate-weinmap-of-a-ridge}, we note that the leading  $\lambda_1$ would likewise be 0 at any such point.  One could also imagine a similar situation in which the dropoff from crest to bottom gets increasingly steep. In such a case, $\lambda_2$ as a function of transverse position would in fact be largest nearest to the bottom. This thought experiment should dispel a na\"{i}ve misunderstanding of the power of a Frangi filter: a high anisotropy measure (and a large structureness measure \vcleanup{define structureness first}) will not in general identify the crests of a ridge-like structure--it only will highlight that such a pixel is on a ridge-like structure at all. In fact, the particular leading eigenvalue calculated in \cref{sec:calculate-weinmap-of-a-ridge} actually increases as $x$ moves farther away from the crest. In general, the anisotropy measure will not necessarily be at a maximum at the crest of the ridge, but instead, somewhere along it.

Similiarly, the vessel we we wish to identify can not be reasonably expected to behave as perfectly as our toy example. There will likely be small aberrations in a ridgelike structure, such as small divots or depressions in an overall ridge-like structure. Of importance in our data set later (\cref{sec:NCS-data-set}), there will be points where we seem to "lose" our ridgelike structure,
but this is simply due to an error in the sample.

Importantly, this formulation does not require $\lambda_1$ to be approximately zero, just that the curvature in the downward direction is much more significant than in the transverse direction.

Also the crest could be really flat (``hangar shaped''), in which case both are around zero. At the crest of the ridge, we would actually expect both $u_1$ and $u_2$ to be around 0, whereas a point somewhere between the crest and the ``foot'' of the ridge to contain the maximum $u_2$. We will fix this issue specifically by casting this as a multiscale problem in \cref{ch:multifrangi}.

Two other ideas that could fix some other discrepancies mentioned above is to identify these ridges on their own, or also where the 'feet are'. We will discuss these ideas in \cref{ch:segmentation}.

\section{Structureness Measure} \label{sec:frangi-structureness}

There is another concern with using the pure ratio $\Am:= \abs{\lambda_1 / \lambda_2}$ as an identifying feature of ridgelike structures apart from the ones listed above. We could have $\abs*{\lambda_2} \gg \abs*{\lambda_1}$ in a relative sense, but still have $\lambda_2 \approx 0$. As a rather extreme example, we should certainly wish to differentiate a point on the surface where $\lambda_2 \approx 10^{-5} $ and $\lambda_1 \approx 10^{-10}$ from another point where $\lambda_2 \approx 10000$ and $\lambda_2 = 0.1$.

A natural fix to differentiate these points is to introduce a ``structureness'' measure to insure that there is in fact significant curvilinear activity at the point in question. Frangi used $\Sm:= \sqrt{(\lambda_1)^2 + (\lambda_2)^2}$, which is in fact the Frobenius norm of the Hessian matrix. Inclusion of the structureness factor with an appropriately chosen parameter $c$ should result in a high vesselness measure only for regions of significant curvilinear content.


\section{The Frangi Vesselness Measure}

Our goal then is to attach a numerical measure to each pixel in the image %(at a particular scale $\sigma$)
that is large when the anisotropy measure $\Am$ and the structureness measure $\Sm$ is sufficiently large.

The form Frangi arrived at in \cref{eq:frangi-vesselness-measure} in which a factor of $\exp(\cdots)$ and $(1 - \exp(\cdots))$ are multiplied together are simply to ensure that the final vesselness measure $\V$ is largest when $\Am$ is small and $\Sm$ is large enough, with rapid decay in other situations.

Frangi further strengthened the filter by adding an additional case to \cref{eq:frangi-vesselness-measure}, ensuring that $\lambda_2$ is not positive. If we are indeed at a curvilinear ridge, then it is a critical point, and we need the second derivative of the surface in the maximal direction to be negative, which hasn't been accounted for as yet in our formulation of $\Am$ and $\Sm$ -- we wish (for our purposes) to only identify when we are finding crests. $\Am$ will still be small and $\Sm$ will still be large however if we identify a ``trough''.

The only perceivable difference is that the maximum normal curvature will be positive--we are at a local minimum in the direction of $u_2$. In situations where we wish to only identify ridges (bright curvilinear structures), we simply exclude all points where there is not a negative curvature in the maximal direction. Conversely, we could only seek to find valleys (dark curvilinear structures) and thus require $\lambda_2 > 0$, and set the vesselness measure to zero when $\lambda_2 < 0$.



\section{Choosing Parameters $\beta$ and $c$}

The parameters $\beta$ and $c$ are meant to scale so that the peaks of the anisotropy factor $\exp(\frac{-\Am^2}{2\beta^2})$ and the structureness factor $(1-\exp\left(\frac{-\Sm^2}{2c^2}\right))$ coincide enough to be statistically significant at highly curvilinear structures, but rapidly decay in areas not associated with curvilinear content. What values of these parameters are appropriate is ultimately dependent on the context of the problem.

For the default value of the structureness parameter $c$, Frangi suggested that half of (the Frobenius norm of the) Hessian matrix is appropriate, simply because the minimum value of $\Sm$ is zero, and its maximum value is exactly the maximum value of the Frobenius norm over the surface.

With this in mind we would like to introduce the scaling factor $\gamma$, so that $ c = \gamma \Smax$. This creates a minor annoyance though: although the anisotropy factor can certainly attain a value of 1, if $c$ is to take this suggested value, the maximum value of the structureness factor is somewhat smaller than 1. In fact,
\begin{equation}
\begin{aligned}
\max\{\V \} &\le \max\left(
\exp\left(\frac{-\Am^2}{2\beta^2}\right)
\right)
\max\left(
\left(1 - \exp\left(\frac{-\Sm^2}{2(\gamma \Smax)^2}\right)\right)
\right) \\
&\le \max\left\{
\left(1 - \exp\left(\frac{-\Sm^2}{2(\gamma \Smax)^2}\right)\right)
\right\} \\
&= 
\left(1 - \exp\left(\frac{-(\Smax)^2}{2(\gamma \Smax)^2}\right)
\right)
= \left(1 - \exp\left(\frac{-1}{2\gamma^2}\right)
\right)
\end{aligned}
\end{equation}

Thus, when $\gamma$ takes the suggested value of $\gamma = 1/2$, the above calculation suggests that
the maximum theoretical value that the Frangi filter could attain for any image is
$ \max \{ \V \} \le 1 - \exp\left( -1 \right) \approx .8647$.
This (among other obvious reasons) certainly justifies Frangi's description of the vesselness measure as only ``probability-like.'' Still, we would like the filter's sensitivity to relative structureness to not have the effect of dampening the Filter as a whole, so we will introduce a rescaling factor $a_\gamma$, which is an explicit function of $\gamma$ that rescales $\V$ so that the structureness factor has a maximum value of 1 regardless of choice of $\gamma$. Our modified Frangi vesselness measure is thus

%    \begin{minipage}{\textwidth} \centering
\begin{gather} \label{eq:V_rescaled}
\V(x_0,y_0) = \begin{cases}
0 & \text{if} \quad \lambda_2 > 0 \\
a_\gamma \exp\left(\frac{-\Am^2}{2\beta^2}\right)
\left(1 - \exp\left(\frac{-\Sm^2}{2(\gamma \Smax)^2}\right)\right) & \text{otherwise}
\end{cases} \\
\shortintertext{where, as before,}
%\begin{equation} \label{frangi-def-anisotropy-structureness-v1}
\Am := \abs*{\frac{\lambda_1}{\lambda_2}}
\;,\;
\Sm := \sqrt{\lambda_1^2 + \lambda_2^2}
\;\textrm{and}\;
a_\gamma = \left(1-\exp\left(\frac{-1}{2\gamma^2}\right)\right)^{-1} \notag\\
%\end{equation}
\shortintertext{and}
%\begin{equation*}
\abs*{\lambda_1} \le \abs*{\lambda_2}
\; \textrm{are eigenvalues of the Hessian matrix at point} \; (x_0, y_0).
%\end{equation*}
\end{gather}
%    \end{minipage}

For $\beta$, Frangi suggested an innocuous intermediate point, $\beta=1/2$ (and thus $2\beta^2 = 1/2$).
As we will show later, choosing the structureness parameter $\gamma$ is rather important for the context especially if the background (non-ridgelike structure) is significant and noisy. $\beta$ should be strengthened/relaxed depending on how dramatic the curvilinear structures are in comparison to the rest of the image. We shall show empirically that contexts in which more 'bloblike' structures are known to be present than that for which the Frangi filter was originally designed, we will benefit from a smaller choice of $\beta$.

Considering as the anisotropy measure $\abs*{\lambda_1 / \lambda_2} \in [0,1]$,  we can actually visualize how much the 
anisotropy factor varies depending on our choice of $\beta$, as seen in \cref{fig:anisotropy-parameter-demo}.

We can theoretically choose any values $0 < \beta, \gamma < \infty$ for the two parameters. Two particular limits are of theoretical interest.

As $\beta \to \infty$, the anisotropy factor tends to $1$, and the Frangi filter will simply return the value of the structureness factor. Similarly, as $\gamma$ tends to $0$, the structureness factor tends to $1$, and the Frangi filter will return the value of the anisotropy factor. We will use this fact in \cref{ch:results-analysis} to provide a visual demonstration of each individual factor. The limits in the other directions are less interesting; if either $\beta \to 0$ or $\gamma \to \infty$, their respective factors will tend to $0$, making the entire filter zero everywhere.

\begin{figure}[t] \centering
  \includegraphics[height=0.4\textheight]{anisotropy_parameter_demo}
  \caption{Dependence of the anisotropy factor on its parameter}
  \label{fig:anisotropy-parameter-demo}
\end{figure}

\begin{figure}[t] \centering
  \includegraphics[height=0.4\textheight]{structureness_parameter_demo}\\[12pt]
  \caption{Dependence of the structureness factor on its parameter}
  \label{fig:structureness-parameter-demo}
\end{figure}

\cref{fig:structureness-parameter-demo} is a similar presentation of the dependence of the structureness kernel on its parameter $\gamma$.
A convenient by-product of our redefinition of $\gamma$ actually is that can group together $\Sm$ with the $\Smax$ that appears in the denominator, allowing us to refer to a structureness ratio $\Sm/\Smax$. In fact, we could perfectly well define the Frangi filter to be in terms of this ratio--though we refrain presently since we acknowledge specific cases where $c$ can be chosen to be fixed independent of a calculated $\Smax$ value.


The ultimate choice of these parameters $\beta$ and $\gamma$ have the overall effect of tuning the selectivity of the Frangi filter itself. In \cref{fig:frangi3d-selection}, we show the shape of our rescaled filter depends on the anisotropy measure $\abs{\lambda_1 / \lambda_2}$ and the ``structureness ratio'' $\Sm/\Smax$ for some illustrative choices of $\beta$ and $\gamma$. Each of these filters has been rescaled with $a_\gamma$. The steepness of the first figure suggests that the filter is in fact, stricter compared to the default parameter selection. \cref{subfig:frangi3d-loose} shows the filters continued strength over a much larger range of values, indicating that it may not work effectively. In particular, choosing $\gamma$, and thus $c$, poorly will lead to quite a lot of noise.
An interesting manifestation of this is that the default implementation of the Frangi filter \autocite{scipy} chooses an seemingly arbitrarily fixed value for the structureness parameter, perhaps causing many na\"ive applications of the Frangi filter to give a needlessly poor result, simply becuase the filter was not correctly parametrized in relation to the image at hand.

\begin{figure}[t] \centering
\subfloat[Strict]{\includegraphics[height=0.3\textheight]{4-rs}} \label{subfig:frangi3d-strict} \\[-.2em]
\subfloat[Standard]{\includegraphics[height=0.3\textheight]{14-rs}} \label{subfig:frangi3d-default} \\[-.2em]
\subfloat[Loose $\gamma$]{\includegraphics[height=0.3\textheight]{12-rs}} \label{subfig:frangi3d-loose} \\[12pt]
\caption{Strictness of Frangi filter based on parametrization}
\label{fig:frangi3d-selection}
\end{figure}

	We now take a quick tangent from our description of the Frangi filter to develop and justify our ``multiscale'' approach.
	

\chapter{Linear Scale Space Theory} \label{sec:scale-space-theory}
    
    Although the ideas presented above require differentiation of continuous
    surfaces, our image is in fact composed of discrete pixels. That is, our previous discussions have been in terms of an image as the continuous surface in \cref{def:image_as_surface},
    rather than the more realistic discrete pixel matrix as in \cref{def:image_as_pixel_matrix}.
    The present section seeks to address this disconnect. Divided differences can serve as an analogue of differentiation in discrete contexts, but our "derivative" and any use of it is then completely dependent on the bias of our limited sampling of the ``true'' 3D surface. Our main goal is to counter against some of the bias of our particular sampling. In particular, we wish to not over-represent structures that are clear at our resolution without giving appropriate weight to larger structures as well. Ideally, we would like statements we make about higher order phenomena to be stable enough at least that we would arrive at the same conclusions from analyzing a higher resolution image of the same surface.
    Koenderink \cite{Koenderink} argued that "any image can be embedded in a one-parameter family of derived images (with resolution as the parameter) in essentially only one unique way" given a few of the so-called scale space axioms. He (and others) showed that a small set of intuitive axioms imply that any such family of images $\{K_\sigma\}$ must satisfy the heat equation
    
    \begin{equation} \label{eq:heat-eq-for-family}
    \begin{dcases}
    \Delta K(x,y;\sigma) = \frac{\partial K}{\partial \sigma}(x,y;\sigma)
    \;\text{for}\;\; \sigma \ge 0  \\
    K(x,y, 0) = u_0(x,y)
    \end{dcases}
    \end{equation}
    
    where $u_0: \R^2 \to \R$ is the original image (viewed as a continuous surface), $\sigma$ is the resolution parameter, and $K: \R^2 \to \R $ for each fixed $\sigma$.
    
    
    This result is intuitive and desirable--we would anticipate a lower-resolution version of our original image to represent a diffusion or ``blurring'' of the initial scales information. Much work has been done to formalize this approach \cite{GSST-book}. This has resulted in various formulations of a minimal set of axioms from which all other desirable properties of the scale space can be derived, culminating eventually in the necessity and sufficiency of \cref{eq:heat-eq-for-family} itself. We will refrain from such an exhaustive development in the present text, but rather list some desirable properties and outline the approach.
	    
    \subsection{Properties/Axioms of Linear Scale Space Theory}
	 To make matters manageable, we require the one-parameter family to be generated by an operator on the original image:
	 
	\begin{equation}
		\left\{\ K(x,y;\sigma) = T_\sigma u_0
		\ \lvert \ 
		\sigma \ge 0
		\; ,\; K(x,y,;0) = u_0
		\ \right\} 
	\end{equation}
    
	The following axioms are then requirements on what sort of operation $T_\sigma$ should be.
		
    \begin{axiom}[Linear-shift and Rotational Invariance]
    	\label{axiom:linear-shift-and-rotation}
    We require that no position in the original signal is favored.  This is intuitive, as our operation should apply to any image fairly, regardless of where content is found in the image, and cropping or rotating our initial image should not affect resolution of the content.
\end{axiom}
   
     \begin{axiom}[Semigroup property] \label{axiom:semigroup}
   	The semigroup property is simply that transforming
   	the original image by some resolution $\sigma$ should
   	have the same overall effect of two successive
   	transformations $\sigma_1$ and $\sigma_2$, i.e.
   	
   	\begin{equation}	
   	T_{\sigma} u = T_{\sigma_1 + \sigma_2} u
   	\end{equation}
   \end{axiom}
 
    \begin{axiom}[Continuity of Scale Parameter]
    	\label{axiom:continuity} There is no reason for the scale parameter to be discrete; we may alter the resolution with whatever precision we desire. That is, we take the resolution
    parameter $\sigma$ to be any nonzero real number. Moreover, we require that the operator behaves continuously with respect to the scale parameter.
        \end{axiom}
    
%    As $\sigma \downarrow 0$, we would like to end up the initial conditions (i.e. so that $T_0 u_0 = u_0$). 	This requirement, given \cref{axiom:semigroup} and an argument from functional analysis (see \cite{hille1957functional}) implies that there exists an infinitesimal generator $A$
%	which is the limit case of our desired operator $T$; that is
%	\begin{equation} \label{eq:infinitesimal-generator}
%	A u_0 = \lim_{\sigma \downarrow 0} \frac{T_\sigma u_0 - u_0}{\sigma}
%	\end{equation}.
	
	 
	
    
  
%     \subsubsection{Linearity of generator}
%        ``Linearity implies that all-scale space properties valid for the original signal will transfer to its derivatives. Hence, there is no commitment to certain aspects of image structure, such as the zero-order representation, or its first- or second-order derivatives.''
   
	The following requirement has great implication, and is also
	very successful in encoding our intuitive sense
	of resolution.
   \begin{axiom}[Causality Condition] \label{axiom:causality}
    The causality condition is the one
    that, as resolution decreases, no finer detail is
    introduced into the image. That is, as the scale
    increases, there will be no creation of local extrema
    that did not exist at a smaller scale.
    \end{axiom}
    To make this more precise, if $K(x_0,y_0 ; \sigma_0)$ (that is, a point $(x_0,y_0)$ at some particular
    resolution $\sigma_0$) is a local maximum at that resolution,
    then an increase in scale cannot make this maximum more prominent, i.e.
    \begin{equation}
    \left\{\begin{aligned}
    \nabla K(x_0,y_0; \sigma_0) &= 0 \\
    \Delta K(x_0,y_0;\sigma_0) &< 0
    \end{aligned}\right.
	\quad \implies \quad
	K(x_0,y_0;\sigma_1) \le K(x_0,y_0;\sigma_0)
	\; \forall\; \sigma_1 \ge \sigma_0
    \end{equation}
    
    Similarly, if $K(x_0,y_0;\sigma_0)$ is a local minimum (with respect to space), then an increase in scale cannot make such a valley more profound, i.e.
   \begin{equation}
   \left\{\begin{aligned}
   \nabla K(x_0,y_0; \sigma_0) &= 0 \\
	\Delta K(x_0,y_0;\sigma_0) &> 0
	\end{aligned}\right.
	\quad \implies \quad
	K(x_0,y_0;\sigma_1) \ge K(x_0,y_0;\sigma_0)
	\; \forall\; \sigma_1 \ge \sigma_0
	\end{equation}
    
    This encodes our intuition that no image feature should be sharpened by a descrease in resolution. The only result is a (non-strictly) monotonic blurring of the image as scale parameter $\sigma$ tends to infinity.

    \subsection{Sufficiency of the Gaussian Kernel}
    
    Although we will omit it here, the above requirements are actually sufficient in proving not only that the operator $T_\sigma$ is a convolution, but that the heat equation described in \cref{eq:heat-eq-for-family} must hold. This has been shown in various ways, both by Koenderink \cite{Koenderink}, Babaud \cite{babaud}, as well as Lindeberg in \cite{GSST-book}. Of course, once \cref{eq:heat-eq-for-family} has been established, it is straightforward to show that
    
    \begin{equation}
       K(x,y;\sigma) = T_{\sigma} u_0 = G_\sigma \star u_0
       	\quad \textrm{where}\quad
       	G_\sigma := \frac{1}{2\pi \sigma^2} e^{\left(-\abs{x}^2 / (2\sigma^2)\right)}
        \end{equation}
    
    is a solution.  That is, the family can be generated by convolution with a Gaussian kernel. Lindeberg and others furthered this by arguing that f$T_\sigma$'s continuity as $\sigma$ approaches zero ultimately implies that convolution by a Gaussian the \textit{unique} operator that generates this scale space. 
%   To this, show that:
%    \begin{itemize}
%    \item a kernel satisfying the above axioms must satisfy the heat equation
%  	\item the gaussian kernel satisfies that.
%    \item gaussian kernel is the only kernel that works.
%    \end{itemize}
%    
%    That is,
%    
%    
%	We can show that this solution solves the heat equation.
%	Given $u_0$ as a continuous image (unscaled),
%	we construct PDE with this as a boundary condition.
%    
%    \begin{equation} \label{heat-eq}
%    u: \R^2 \supset \Omega \to \R \; \textrm{with} \; u(\bm{x},t) : \;
%    \begin{cases}
%    \frac{\partial u}{\partial t} (\bm{x}, t) = \Delta u(\bm{x},t) & ,\; t \ge 0 \\
%    u(\bm{x},0) = u_0(\bm{x}) 
%    \end{cases}
%    \end{equation}
%    
%    We show that
%    \begin{equation}
%    u(\bm{x},t) = \left(G_{\sqrt{2t}} \star u_0 \right)(\bm{x})
%    \end{equation}
%    solves \cref{heat-eq}.
%
%    First, we need a quick lemma regarding differentiation a continuous convolution.
%    \begin{lemma} \label{dconvolution}
%    	Derivative of a convolution is the way that it is (obviously rewrite this).
%    \end{lemma}
%    \begin{proof}
%    	For a single variable,
%    	\begin{align}
%    	\frac{\partial}{\partial \alpha} \left[ f(\alpha) \star g(\alpha) \right]
%    	&= \frac{\partial}{\partial \alpha} \left[ 
%    	\int f(t) g(\alpha - t) dt \right] \\
%    	&=  \int f(t) \frac{\partial}{\partial \alpha}\left[ g(\alpha - t)  \right] dt \\
%    	&=  \int f(t) \left(\frac{\partial g}{\partial \alpha}\right) g(\alpha - t) dt \\
%    	&=  f(\alpha) \star g'(\alpha)
%    	\end{align}
%    	By symmetry of convolution we can also conclude 
%    	\[\frac{\partial}{\partial \alpha} \left[ f(\alpha) \star g(\alpha) \right]
%    	= f'(\alpha) \star g(\alpha)
%    	\]
%    	
%    	If $f$ and $g$ are twice differentiable, we can compound this result to show a similar statement holds for second derivatives, and then, given the additivity of convolution,
%    	we may conclude
%    	\begin{equation}
%    	\Delta \left(f \star g \right) = \Delta(f) \star g = f \star \Delta(g) 
%    	\end{equation} 
%    \end{proof}
%    \begin{theorem}
%    	$u(\bm{x},t) = \left(G_{\sqrt{2t}} \star u_0 \right)(\bm{x})$ solves the heat equation.
%    \end{theorem}
%    \begin{proof}
%    	We focus on the particular kernel
%    	\[
%    	G_{\sqrt{2t}} = \frac{1}{4\pi t} e^{\left(-\abs{x}^2 / (4t)\right)}
%    	\]
%    	
%    	
%    	Then
%    	\begin{align}
%    	\frac{\partial u}{\partial t} (\bm{x}, t)
%    	&= \frac{\partial}{\partial t} \left(G_{\sqrt{2t}}(\bm{x},t) \star u_0(\bm{x})\right)  \\
%    	&= \frac{\partial}{\partial t} \left(G_{\sqrt{2t}}(\bm{x},t)\right) \star u_0(\bm{x})  \\
%    	&= \frac{\partial}{\partial t} \left(
%    	\frac{1}{4\pi t} e^{\left(-\abs{x}^2 / (4t)\right)} \right) \star u_0(\bm{x}) \\
%    	&= \left[
%    	-\frac{1}{4\pi t^2} e^{\left(-\abs{x}^2 / (4t)\right)}
%    	+ \frac{1}{4\pi t}\left(\frac{-\abs{x}^2}{4t^2}\right) e^{-\abs{x}^2 / (4t)}
%    	\right] \star u_0(\bm{x}) \\
%    	&= -\frac{1}{4t^2} \left( e^{\left(-\abs{x}^2 / (4t)\right)} 
%    	+ \abs{\bm{x}}^2 G_{\sqrt{2t}}(\bm{x},t)
%    	\right) \star u_0(\bm{x})
%    	\end{align}
%    	and from the previous lemma,
%    	\[
%    	\Delta u(\bm{x}, t) = \Delta\left( G_{\sqrt{2t}} \star u_0(\bm{x})\right)
%    	= \Delta\left( G_{\sqrt{2t}} \right)\star u_0(\bm{x})
%    	\]
%    	
%    	We explicitly calculate the Laplacian of $G_{\sigma}(x,y) = A \exp(-\frac{x^2 + y^2}{2\sigma^2})$ as follows:
%    	
%    	\begin{align*}
%    	\frac{\partial}{\partial x} G_{\sigma}(x,y)
%    	&= A \left( \frac{-2x}{2\sigma^2}\right) \exp\left(-\frac{x^2 + y^2}{2\sigma^2}\right) \\
%    	\implies \frac{\partial^2}{\partial^2 x} G_{\sigma}(x,y)
%    	&= A \cdot \frac{\partial}{\partial x}
%    	\left[ - \frac{x}{\sigma^2} \exp\left(-\frac{x^2 + y^2}{2\sigma^2}\right) \right] \\
%    	&= A \left[ - \frac{1}{\sigma^2} \exp\left(-\frac{x^2 + y^2}{2\sigma^2}\right) 
%    	+ \frac{x}{\sigma^2} \cdot \frac{2x}{2\sigma^2} \exp\left(-\frac{x^2 + y^2}{2\sigma^2}\right) \right] \\
%    	&= A \exp\left(-\frac{x^2 + y^2}{2\sigma^2}\right)
%    	\left[ - \frac{1}{\sigma^2} + \frac{x^2}{\sigma^4} \right] \\
%    	&= \frac{1}{\sigma^2} G_\sigma(x,y)  \left[ \frac{x^2}{\sigma^2} - 1\right]
%    	\end{align*}
%    	
%    	By symmetry of argument we also may conclude
%    	\[
%    	\frac{\partial^2}{\partial y^2} G_{\sigma}(x,y) = \frac{1}{\sigma^2} G_\sigma(x,y)  \left[ \frac{y^2}{\sigma^2} - 1\right]
%    	\]
%    	
%    	and so
%    	
%    	\begin{equation}
%    	\Delta G_\sigma(x,y) =
%    	\frac{\partial^2}{\partial x^2} \left(G_{\sigma}\right)
%    	+ \frac{\partial^2}{\partial y^2} \left(G_{\sigma}\right)
%    	= \frac{1}{\sigma^2} G_\sigma(x,y) \left[ \frac{x^2 + y^2}{\sigma^2} - 2\right] 
%    	\end{equation}
%    	Then, given \cref{dconvolution}, we conclude
%    	\begin{equation}
%    	\Delta \left[ G_\sigma(x,y) \star u_0(x,y) \right] 
%    	= \left(\frac{1}{\sigma^2} G_\sigma(x,y) \left[ \frac{x^2 + y^2}{\sigma^2} - 2\right]\right) \star u_0(x,y)
%    	\end{equation}
%    	
%    	For particular choices of $\sigma(t) = \sqrt{2t}$ and $A = \frac{1}{4\pi t}$,
%    	we see 
%    	\begin{align}
%    	\Delta \left[ G_{\sqrt{2t}}(x,y) \star u_0(x,y) \right] 
%    	&= \left(\frac{1}{2t} G_{\sqrt{2t}}(x,y) \left[ \frac{x^2 + y^2}{2t} - 2\right]\right) \star u_0(x,y) \\
%    	&= \left(G_{\sqrt{2t}}(x,y) \left[ \frac{x^2 + y^2}{4t^2} - \frac{1}{t}\right]\right) \star u_0(x,y)
%    	\end{align}
%    	We then calculate the time derivative,
%    	using our particular choice of $\sigma(t) = \sqrt{2t}$ and $A = \frac{1}{4\pi t}$ as:
%    	
%    	\begin{align}
%    	\frac{\partial}{\partial t} \left[ G_{\sigma(t)}(x,y) \star u_0(x,y) \right]
%    	&= \frac{\partial}{\partial t} \left[ G_{\sigma(t)}(x,y) \right] \star u_0(x,y) \\
%    	&= \frac{\partial}{\partial t} \left[ G_{\sqrt{2t}}(x,y)\right] \star u_0(x,y) \\
%    	&= \frac{\partial}{\partial t} \left[
%    	\frac{1}{4\pi t} \exp\left(-\frac{x^2 + y^2}{4t}\right) \right] \star u_0(x,y) \\
%    	&= \left[ -\frac{1}{4\pi t^2} \exp\left(-\frac{x^2 + y^2}{4t}\right) + 
%    	\frac{1}{4\pi t}\left( \frac{x^2 + y^2}{4t^2} \exp\left(-\frac{x^2 + y^2}{4t}\right)\right)
%    	\right] \star u_0(x,y) \\
%    	&= \left(G_{\sqrt{2t}}(x,y) \left[ \frac{x^2 + y^2}{4t^2} -\frac{1}{t}\right]\right) \star u_0(x,y)
%    	\end{align}
%    	
%    	Combining these results, we find that
%    	\begin{equation}
%    	\frac{\partial}{\partial t} \left[ G_{\sqrt{2t}} \star u_0 \right]
%    	= \Delta \left[ G_{\sqrt{2t}} \star u_0 \right] 
%    	\end{equation}
%    	
%    	as desired. \end{proof}
%    
    
    
    \subsection{Scale Spaces over Discrete Structures} \label{subsec:discrete-scale-space}
    
    The above developments from scale space axioms have (since their first appearance)
    been recast in terms of discrete structures (rather than continuous surfaces) as in \cite{lindeberg-discrete}. However, we've chosen to present the above in their original continuous surface for clarity of argument. The discrete case is not much different--
    we still have the same axioms, and it can be shown that the family of scaled images
    must simply satisfy a discrete version of the heat equation.
    However, viewing our actual image
    \cref{def:image_as_pixel_matrix} as a sample of a continuous
    surface \cref{def:image_as_surface},
    we might  expect our convolution by the Gaussian to ``commute'' with our supposed sampling of the continuous signal,
    or even that we could simply convolve our discrete signal with a discretely sampled Gaussian kernel. The latter in fact, seems to be an often implemented interpretation of scale space theory.
    
    To be clear, the ``sampled'' 1D Gaussian Kernel we have in mind might be given by:
   
    \begin{defn}[Sampled Gaussian Kernel and Generated Family]
    	\[
    	g(n ; \sigma) = \frac{1}{2\pi \sigma} e^{-n^2 / 2\sigma} \;,\quad -\infty < n < \infty
    	\]
    	\end{defn}
    and the  resulting (1D) convolution would be given by
    \[
	    K(x,\sigma) = \sum_{n=-\infty}^{\infty} g(n;\sigma) f(x-n)
	    \quad \textrm{for} \quad x \in \Z, \; \sigma > 0
    \]
    In \cite{lindeberg-discrete} and in particular \cite{lindeberg1988-discreteconstruction}, Lindeberg demonstrated that the sampled Gaussian kernel violates not only semigroup property (\cref{axiom:semigroup}), but--much less forgivably--the causality property (\cref{axiom:semigroup}). There is absolutely no guarantee that convolution with a sampled Gaussian kernel will not create ``spurious'' structures as resolution increases.
    
    
    Fortunately, Lindeberg was immediately able to remedy this by providing a discrete analogue of the Gaussian kernel, which does satisfy \cref{axiom:causality} and \cref{axiom:semigroup}:
    
    \begin{defn}[Discrete Gaussian Kernel]
    	The discrete Gaussian kernel, which can be shown to be a suitable generator
    	for scale space, is given by
    	\begin{equation}
    	T(n;\sigma) = e^{-\alpha \sigma} I_n(\alpha \sigma) ,\quad\,
    	 I_n(\sigma) = I_{-n}(\sigma) = (-1)^n J_n(i\sigma) 
    	 \quad n \ge 0 , \sigma,\alpha > 0
    	 \end{equation}
    \end{defn}
    where $I_n$ are the modified Bessel functions of integer order based on the
    ordinary Bessel functions $J_n$, i.e.
    \[
    I_n(x) = \sum_{m=0}^{\infty} \frac{1}{m! (m+n)!}
	    	\left(\frac{x}{2}\right)^{2m+n} \;,\quad n \ge 0
    \]
    where we have taken the liberty of simplifying the typical definition \cite{abramowitz-stegun} (which involves the gamma function), since we only desire
    Bessel functions of integer order. The parameter $\alpha$ above is simply an
    optional scaling parameter which is simply set to $1$ hereforth.
    
    The derived family of 1D signals is then given by
\begin{equation} \label{eq:derived-family-from-discrete}
        K(x,\sigma) = \sum_{n=-\infty}^{\infty} T(n;t) f(x-n)
        \quad \textrm{for} \quad x \in \Z, \; t > 0
        \end{equation}
    
    The compatibility of scale space theory and derivatives on discrete structures and     extension to two dimensions was also demonstrated by Lindeberg in \cite{lindeberg-discrete-derivative} and \cite{lindeberg1998feature}. In particular, we may take derivatives of the convolutions of our discrete images
    using, say, a local central difference.
    Lastly, the 2D version of the family given in \cref{eq:derived-family-from-discrete} can be obtained by independent convolution of its dimensions (i.e. it is separable). We will make these
    ideas explicit in \cref{ch:implementations}.
 \chapter{The Multiscale and Signed Frangi Filter} \label{ch:multifrangi}
    
     With the ideas of scale established, we may return to our discussion of the Frangi filter.
    Our ideas of scale developed in the previous section imply that, if the ridgelike structures we wish to detect are more prominent at different scales, then a multiscale approach is the natural one. Considering the dependence of the Frangi filter's response on choice of scale (as demonstrated in \cref{sec:frangi}), we wish to probe at multiple scales
    regions that would receive a high vesselness score at any range and somehow merge the result. Frangi \cite{frangi-paper} approached this problem by simply taking the maximum vesselness measure over all scales. Thus the multiscale Frangi vesselness score at the pixel $(x_0, y_0$) would be 
    
    \begin{equation} \label{eq:Vmax}
    \Vmax(x_0, y_0) =
    	\underset{\sigma \in \Sigma}{\max}\left\{  \Vsigma (x_0, y_0) \right\}
    \end{equation}
    
    where $\Sigma := \left\{ \sigma_0, \sigma_1 , \cdots, \sigma_N \right\}$ is
    the set of scales at which to probe, and \Vsigma is the Frangi vesselness measure at scale $\sigma$ for the pixel $(x_0,y_0)$. The set of scales $\Sigma$ should be chosen to be representative enough of all scales where meaningful content is expected to be found.
    
   
    \section{Rudimentary Thresholding}
    
    After the maximization in \cref{eq:Vmax}, we are left with a matrix with as many pixels as the original image, all with a vesselness measure between $0$ and $1$ for each pixel in the image.
         
    At this point, Frangi \cite{frangi-paper} refrained from explicitly interpreting the score assigned by \cref{eq:Vmax}; that is--whether a particular pixel $(x_0,y_0)$ in the image definitely respresents a vessel or not based on its Frangi score. Instead, he cautioned that the result should not be used as a segmentation method alone, and moreover that the width of the vasculature cannot be determined rigorously from the Frangi filter (as discussed in \cref{ch:unifrangi}).
   
    Nonetheless, we wish to demonstrate the usefulness of the Frangi filter within our image domain towards segmentation. We should at least expect that a well-tuned Frangi filter (on an appropriately registered and denoised sample) should assign its highest scores to vessel pixels. We can select these strong Frangi responses and use them as seeds for some subsequent algorithm. A straightforward enough approach would be to simply threshold at some fixed value $\alpha$. Such thresholding was used in \cite{huynh2013filter}. 
   

   \begin{equation}	    
      {\VSigma}_{\alpha}(x_0,y_0) = \begin{cases}
   1 & \textrm{if}\quad \Vmax(x_0,y_0) \; \ge\;  \alpha \\
   0 & \textrm{else}
  \end{cases}
   \quad , \; \alpha > 0
   \; \textrm{for } \; \alpha \;\textrm{fixed}.
   \end{equation}
 

If we insist on such a performing such a thresholding, the ``correct'' choice of $\alpha$ unfortunately seems to depend on the image domain, so user intervention when dealing with the problem domain seems to be the best strategy. We would hope that some normalization of our data set would permit a single choice of $\alpha$ across all samples, but unfortunately we cannot guarantee this. Without prior knowledge of an appropriate choice of $\alpha$, we may have to simply select $\alpha$ by trial and error.


A good alternative method of thresholding would be to simply select the highest scores from each responses: we calculate a high percentile score and threshold at that value. Due to the large number of zeros outputted by the filter, we opt instead to take the $q$th percentile of only nonzero values of $V_\Sigma$. 
We briefly demonstrate this in \cref{fig:qthresh_demo} on a particularly well-behaved sample. The top left value image is the base image, the top right is $\Vmax$, the bottom left is $\Vmax$ thresholded at the 95th percentile, and the bottom left is thresholded at the 98th percentile.

\begin{figure} \centering
\subfloat{\includegraphics[width=0.48\linewidth]{qthresh_demo_img}}
\subfloat{\includegraphics[width=0.48\linewidth]{qthresh_demo_Fmax}} \\
\subfloat{\includegraphics[width=0.48\linewidth]{qthresh_demo_q95}}
\subfloat{\includegraphics[width=0.48\linewidth]{qthresh_demo_q98}} \\
\caption{Nonzero-percentile thresholding of \Vmax (95th and 98th percentile)}.
\label{fig:qthresh_demo}
\end{figure}

Of course, the downside of this method is that we do not know in general the size of the network, and  we may miss entire branches of the vascular network if there is a large amount of curvilinear content elsewhere in the image.

We will discuss alternatives methods of aggregating results from our multiscale method, as well as optimal values for parameters and scales
in \cref{ch:segmentation}. As a final note, we admit that any future extensions of this work (as will be discussed in \cref{ch:conclusion}) need not proceed from either of these thresholdings; analyzing the raw vesselness score \Vmax, or even the un-merged scale-wise scores may prove far more rewarding.

\section{The Signed Frangi Filter} \label{sec:signed-frangi-filter}
We finally introduce the novel (yet straightforward) notion of the signed Frangi filter. As will be shown in \cref{ch:segmentation}, we can befefit from simultaneously calculate for a dark background and a light background. Since the Frangi filter normally throws away any response where $\lambda_2 < 0$ (if dark curvilinear features are targeted) or $\lambda_2 >0$ (if light curvilinear features are targeted), we lose no computation time at all (although we must store more results). After computing the multiscale result, we can easily separate these into a positive and a negative strain, which we will denote
$\Vmax^{(+)}$ and $\Vmax^{(-)}$. Our $\Vmax^{(+)}$ is the same as our $\Vmax$ before, and $\Vmax^{(-)}$ is the same result as if we had taken the Frangi filter while only looking for the opposite type (light/dark) curvilinear feature. Plotting $\Vmax$ over a scale of $[-1,1]$ demonstrates an interesting effect. Whereas the Frangi filter generally is not reliable in terms of accurately predicting widths of through-like (or ridge-like) curvilinear features, by somehow combining the results of our positive and negative strains. We \textit{can} get a sense of the width by looking at where there is a relatively strong response of opposite sign.


All that remains to describe mathematically is how to actually calculate the derivatives of our images and deal with the ultimately discrete nature of our samples.    
\chapter{FFT-based Discrete Derivatives}

According to \cref{subsec:discrete-scale-space}, we may calculate derivatives of our structure by calculating a gradient on our convolved image. Our method of calculating the gradient of a matrix uses a second-order accurate central difference, as in \cite{fornberg-1988}.

We note in passing that we may take the derivative of the Gaussian kernel and then convolve it, and the effect will be the same as if we had taken the derivative subsequently \cite{DIPGW}. This could offer some computational speedup if we wish to run this procedure on many samples and fixed scale sizes. For the time being, we will convolve first, although our method will differ from standard convolution.

%Given that we are taking a derivative of a convolution, we first show that these operations commute--that is, we may actually take the derivative of the convolution kernel OR the function itself, and then convolve, and the result should be equivalent to taking the derivative of
%convolving 

%Note the following 6 methods should all theoretically be equivalent:
%
%\begin{itemize}
%   	\item convolve discrete image with a gaussian kernel, then take derivatives (no FFT). This is the ``classical approach''
%   	\item Take derivatives of gaussian kernel, then convolve with the image/signal (again, no FFT)(due to \vcleanup{cite theorem})
%   	\item FFT image and FFT gaussian kernel then convolve in freq. space, then IFFT, then take derivatives in xy-space (my implementation)
%   	\item take derivatives of gaussian kernel, then FFT it,  FFT the image, then colvolve in frequency space, then  IFFT (slower due to size of large kernels)
%   	\item FFT image, then multiply by theoretical gaussian kernel in freq. space, then IFFT, then take derivatives in xy-space.
%   	\item FFT image, then multiply by theoretical gaussian kernel in freq space, then take derivatives in freq space, then IFFT.
%   	
%   	
%\end{itemize}


\section{Fourier Transforms}

In practice, the convolutions described above  are very slow for large scales ($\large \sigma$), as the size of the kernel is very large. Instead, we will perform a fast Fourier transform, which requires only $\mathscr{O}\left(N\cdot \log_2N\right)$ operations for a one dimension signal of length $N$, as compared to the $N^2$ operations required of a conventional discrete Fourier transform \cite{DIPGW}. We will briefly outline the theory of Fourier transforms.

\subsection{Fourier Transform of a Continuous 1D Signal}

\vcomment{start with 1D but then extend/rewrite}

A periodic signal (real valued function) $f(t)$ of period $T$ can \vcomment{justify?} be expanded in an infinite basis as follows:

\begin{equation}
f(t) = \sum_{-\infty}^{\infty} c_n e^{i\frac{2\pi n}{T}t} \;,\quad
	c_n = \frac{1}{T}\int_{-T/2}^{T/2} f(t) e^{-i\frac{2\pi n}{T}t} dt
	\end{equation}

The Fourier transform of a 1D continuous function is defined by
\begin{equation} \label{1D-CFT}
F(\mu) := \FT\{f(t)\} \;=\; \int_{-\infty}^{\infty} f(t) e^{i2\pi \mu } dt
\end{equation}

An inverse transform will then recover our original signal:
\begin{equation} \label{1D-ICFT}
f(t) = \FT^{-1}\left\{F(\mu)\right\} = \int_{-\infty}^{\infty} F(\mu) e^{i2\pi \mu t} dt
\end{equation}

Together, \cref{1D-CFT} and \cref{1D-ICFT} are referred to as the \textit{Fourier transform pair} of the signal $f(t)$. 

\subsection{Fourier Transform of a Discrete 1D signal}

We wish to develop the Fourier transform pair for a discrete signal., following \cite{DIPGW}. We frame the situation
as follows: a continuous function $f(t)$ is represented as the sampled function $\tilde{f}(t)$ by multiplying it by a sampling (or impulse) function, an infinite series of discrete impulses with equal spacing $\Delta T$:

\begin{equation} \label{1D-sampling-function}
s_{\Delta T}(t) := \sum_{n=-\infty}^{\infty} \delta[t - n\Delta T] \;,\quad
\delta[t] = \begin{cases} 1 \;,\; & t=0 \\ 0 \;,\;& t \ne 0 \end{cases}
\end{equation}

where $\delta[t]$ is the discrete unit impulse.

The discrete sample $f(t)$ is then constructed from $f(t)$ by
\begin{equation} \label{1D-discrete-sample}
\tilde{f}(t) = f(t) s_{\Delta T}(t)
\end{equation}

From this we can calculate $\tilde{F}(t)$.
Given the discrete signal $\tilde{f}$, we construct the transform
$\tilde{F}(\mu) = \FT\{\tilde{f}(t)\}$. by expanding $\tilde{f}$ in the same infinite basis as the continuous case.
\begin{equation} \label{1D-DFT-transform}
\tilde{F}(\mu) = \sum_{n=-\infty}^{\infty} f_n e^{-i 2\pi \mu n \Delta T} \;,\quad
f_n = \tilde{f}(n) = f(n\Delta T)
\end{equation}

The transform is a continuous function with period $1 / \Delta T$. 

%\begin{itemize}
%	\item \vtodo{find image processing papers that find hessian from FFT / who uses this?}
%	\item \vtodo{with above: downsides?}
%	\item \vtodo{side by side comparison in a toy example and/or a real problem?}
%\end{itemize}

\subsection{2D DFT Convolution Theorem}\vcomment{the following was adapted in a large part from DFT: an owner's manual. cite? DIP-DW just proves the continous version (in 1D) and then asserts that it works for discrete variables too.}

\vcleanup{get consistent notation--either have the discrete signals be notated as  $\tilde{f}(x,y)$ or $f[x,y]$ or instead comment that it's understood}
\begin{theorem}[2D DFT Convolution Theorem] 
	\vcomment{develop the 2-D DFT from Sec. 4.3 4.4 from DIP-GW (see p235).}
Given two discrete functions are sequences with the same length.
%\vcomment{If they're not actually the same length, DIP-GW suggests to make the final length at least $P = A+C-1$ and $Q = B+D-1$ in the case that the sizes are $A\times B$ and $C\times D$ for $f(x,y)$ and  $h(x,y)$ respectively. Not sure if that matters.}
$f(x,y)$ and $h(x,y)$ for integers $0 < x < M$ and $0 < y < N$, we can take the discrete fourier transform (DFT) of each, where $\mathcal{D}\{\cdots\}$ denotes the DFT.
\begin{align}
F(u,v) := \mathcal{D}\{f(x,y)\} &=
				\sum_{x=0}^{M-1} \sum_{y=0}^{N-1} f(x,y)
				e^{-2\pi i \left(\frac{ux}{M} + \frac{vy}{N}\right)} \\
H(u,v) := \mathcal{D}\{h(x,y)\} &=
				\sum_{x=0}^{M-1} \sum_{y=0}^{N-1} h(x,y)
				e^{-2\pi i \left(\frac{ux}{M} + \frac{vy}{N}\right)}
\end{align}

and given the convolution of the two functions
\begin{equation}
\left(f \star h\right)(x,y) = \sum_{m=0}^{M-1} \sum_{n=0}^{N-1} f(m,n)h(x-m,y-n)
\end{equation}

then $\left(f \star h\right)(x,y)$ and $MN\cdot F(u,v)H(u,v)$ are transform pairs, i.e.
\begin{equation}
\left(f \star h\right)(x,y) = \mathcal{D}^{-1}\left\{MN\cdot F(u,v)H(u,v)\right\}
\end{equation}
\end{theorem}

The proof follows from the definition of convolution, substituting in the inverse-DFT of $f$ and $h$, and then rearrangement of finite sums.

\begin{proof}
\begin{align}
\left(f \star h\right)(x,y) &= \sum_{m=0}^{M-1} \sum_{n=0}^{N-1} f(m,n)h(x-m,y-n) \\
&= \sum_{m=0}^{M-1} \sum_{n=0}^{N-1}
\left(\sum_{p=0}^{M-1} \sum_{q=0}^{N-1} F(p,q)
	e^{2\pi i \left(\frac{mp}{M} + \frac{nq}{N}\right)}\right)
	\left(\sum_{u=0}^{M-1} \sum_{v=0}^{N-1} H(u,v)
	e^{2\pi i \left(\frac{u(x-m)}{M} + \frac{v(y-n)}{N}\right)} \right) \\
&= \left(\sum_{u=0}^{M-1} \sum_{v=0}^{N-1} H(u,v)
	e^{2\pi i \left(\frac{ux}{M} + \frac{vy}{N}\right)}\right)
	\left(\sum_{p=0}^{M-1} \sum_{q=0}^{N-1} F(p,q)
	\left(\sum_{m=0}^{M-1} e^{2\pi i \left(\frac{m(p-u)}{M}\right)}\right)
	\left(\sum_{n=0}^{N-1} e^{2\pi i \left(\frac{n(q-v)}{N}\right)}\right)\right) \\
	&= \left(\sum_{u=0}^{M-1} \sum_{v=0}^{N-1} H(u,v)
	e^{2\pi i \left(\frac{ux}{M} + \frac{vy}{N}\right)}\right)
	\left(\sum_{p=0}^{M-1} \sum_{q=0}^{N-1} F(p,q)
	\left( M \cdot \hat{\delta}_M(p-u) \right)
	\left( N \cdot \hat{\delta}_M(q-v)\right)\right) \\
	&= \left(\sum_{u=0}^{M-1} \sum_{v=0}^{N-1} H(u,v)
	e^{2\pi i \left(\frac{ux}{M} + \frac{vy}{N}\right)}\right)
	\cdot M N F(u,v) \\
	&=MN \cdot \sum_{u=0}^{M-1} \sum_{v=0}^{N-1} F(u,v) H(u,v)
	e^{2\pi i \left(\frac{ux}{M} + \frac{vy}{N}\right)} \\
	&= MN \cdot \mathcal{D}^{-1}\left\{ FH\right\}
\end{align}

where
\begin{equation} \label{delta_multiple}
	\hat{\delta}_N (k) = \begin{cases}
		1 & \text{when } k = 0 \mod N \\
		0 & \text{else}
		\end{cases}
\end{equation}
\end{proof}
Above, we make use of the following lemma \vcomment{add this before DFT convolution theorem and embed the definition of $\hat{\delta}_N$ inside}
\begin{lemma}
Let $j$ and $k$ be integers and let $N$ be a positive integer. Then
\begin{equation} \label{dft_conv_lemma}
\sum_{n=0}^{N-1} e^{2\pi i\left(\frac{n(j-k)}{N}\right)} =  N \cdot \hat{\delta}_N(j-k) 
\end{equation}
\end{lemma}



















\begin{proof}

Consider the complex number $e^{2\pi i (j-k)/N}$. Note first that this is an $N$-th root of unity, since
\[
\left(e^{2\pi i (j-k)/N}\right)^N = e^{2\pi i (j-k)} = \left(e^{2\pi i}\right)^{(j-k)}
= 1^{(j-k)} = 1
\]

In other words, $e^{2\pi i n(j-k)/N}$ is a root of $z^N -1 = 0$, which we can factor as
\begin{equation}
z^N -1 \;=\; (z-1)\left(z^{n-1} + \cdots + z + 1\right) \;=\; (z-1)\sum_{n=0}^{N-1} z^n .
\end{equation}

thus giving us
\begin{equation} \label{dft_conv_lemma_factors}
0 = \left(e^{2\pi i(j-k)/N} - 1\right) \sum_{n=0}^{N-1} e^{2\pi i n(j-k)/N}
\end{equation}

To prove the claim in \cref{dft_conv_lemma}, we consider two cases: First, if $j-k$ is a multiple of $N$, we of course have $e^{2\pi i n(j-k)/N} = \left(e^{2\pi i}\right)^{n(j-k)/N} = 1$  and thus the left side of \cref{dft_conv_lemma} reduces to 
\[
\sum_{n=0}^{N-1} \left(e^{2\pi i}\right)^{n(j-k)/N} = \sum_{n=0}^{N-1} \left(1\right) = N
\].

In the case that $j-k$ is \textit{not} a multiple of $N$, we refer to \cref{dft_conv_lemma_factors}.
The first factor is not zero since, $\left(e^{2\pi i (j-k)/N}\right) \ne 1$ (simply since $(j-k)/N$ is not an integer), and thus it must be that the second factor is 0:
\[
\sum_{n=0}^{N-1} \left(e^{2\pi i (j-k)/N}\right)^n = 0
\]

We can combine these two cases by invoking the definition of \cref{delta_multiple}, giving us the result.
\end{proof}
		
\section{The Fast Fourier Transform}
\vcomment{use DIP-GW p298}
As noted, the above result applies to the Discrete Fourier Transform. We actually achieve a convolution speedup using a Fast Fourier Transform (FFT) instead. We follow the developments of \cite{DIPGW}.
For clarity, we present the following theorems which allow a framework to calculate a 2D Fourier transforms quickly.


First, a 2D DFT may actually be calculated via two successive 1D DFTs, which can be
seen through a basic rearrangement, as follows:

\begin{align} \label{eq:2D-dft-rearrangement}
F(\mu,\nu) &= \sum_{x=0}^{M-1} \sum_{y=0}^{N-1} f(x,y) e^{-i2\pi \left(\mu x/M + \nu y/N\right)} \\
&= \sum_{x=0}^{M-1} e^{-i2\pi \mu x/M} \left[ \sum_{y=0}^{N-1} f(x,y)e^{-i2\pi \nu y/N} \right] \\
&= \sum_{x=0}^{M-1} e^{-i2\pi \mu x/M} \FT_x\{ f(x,y)\} \\
&= \FT_y\{\FT_x \{f(x,y)\} \}
\end{align}

where $\FT_{x'}$ refers to the 1D discrete Fourier transform of the function with respect to
the variable $x'$ only.

Thus, to calculate the Fourier transform $F(u,v)$ at the point $u,v$
requires the computation of the transform of length $N$ for each iterated point $x \in 0,\cdots,M-1$. Thus there are $MN$ complex multiplications and $(M-1)(N-1)$ complex additions in this sequence required for each point $u,v$ that needs to be calculated. Overall, for all points that need to be calculated, the total order of calculations is on the order of $(MN)^2$ We'll also mention that the values of $e^{-i2\pi m/n}$ can be provided by a lookup table rather than ad-hoc calculation.

We now show that a considerable speedup can be achieved through elimination of redundant calculations. In particular, we wish to show that the calculation of a 1D DFT of signal length $M=2^n, n \in \Zpos$ can be reduced to calculating two half-length transforms and an additional $M/2 = 2^{n-1}$ calculations.

\vcomment{we follow DIP-GW variable conventions, which I think are dumb}

To simplify our notation we will use a new notation for the Fourier kernels/basis functions.
Let the 1D Fourier transform be given by

\begin{equation} \label{FFT-defineW}
F(u) = \sum_{x=0}^{M-1} f(x) W_M^{ux},\quad \textrm{where} \quad W_m := e^{-i2\pi/m}
\end{equation} 

We'll define $K \in \Zpos : 2K = M = 2^n$ (i.e. $K = 2^{n-1})$.

We use this to rewrite the series in \cref{FFT-defineW} and split it into odd and even entries in the summation

\begin{align}
F(u) &= \sum_{x=0}^{2K-1} f(x) W_{2K}^{ux} \\
&= \sum_{x=0}^{K-1} f(2x) W_{2K}^{u(2x)}
 + \sum_{x=0}^{K-1} f(2x+1) W_{2K}^{u(2x+1)} \label{FFT-oddevensplit}
\end{align}

We'll get a few identities out of the way (where $m, n, x \in \Zpos$ arbitrary).

\vcomment{this fixes an issue in DIP-GW where the identities were provided in terms of $M$ instead of arbitrary $m$, where the proof uses the results for some value other than $M$ anyway}
\begin{gather} \label{fft-kernelidentities}
W_{(2m)}^{(2n)} = e^{\frac{-i2\pi(2m)}{2m}} = e^{\frac{-i2\pi m}{n}} = W_{m}^{n} \\
W_{m}^{(u+m)x} = e^{\frac{-i2\pi(u+m)x}{m} } = e^{\frac{-i2\pi unx}{m}} e^{\frac{-i2\pi mx}{m}}
			= e^{\frac{-i2\pi ux }{m}} (1) = W_m^{ux} \\
W_{2m}^{(u+m)} = e^{\frac{-i2\pi(u+m)}{2m}} = e^{\frac{-i2\pi ux}{2m}} e^{-i\pi}
	 = W_{2m}^{u} e^{-i\pi} = - W_{2m}^{u}
\end{gather}

Thus we can rewrite \cref{FFT-oddevensplit} as
\begin{align}
F(u)  &= \sum_{x=0}^{K-1} f(2x) W_{2K}^{2ux} + \sum_{x=0}^{K-1} f(2x+1) W_{2K}^{2ux} W_{2K}^{u} \\
\Longrightarrow \quad F(u) &= \left(\sum_{x=0}^{K-1} f(2x) W_{K}^{ux}\right)
+ \left(\sum_{x=0}^{K-1} f(2x+1) W_{K}^{ux}\right) W_{2K}^{u}
\label{fft-oddeven-parens}
\end{align} 

The major advance comes via using the identities \cref{fft-kernelidentities} \vcleanup{fix multitag} to consider the Fourier transform $K$ 
frequencies later \vcomment{wording?}:
\begin{align}
F(u+K) &= \left(\sum_{x=0}^{K-1} f(2x) W_{K}^{(u+K)x}\right)
+ \left(\sum_{x=0}^{K-1} f(2x+1) W_{K}^{(u+K)x}\right) W_{2K}^{(u+K)}\\
\Longrightarrow F(u+K) &= \left(\sum_{x=0}^{K-1} f(2x) W_{K}^{ux}\right)-\left(\sum_{x=0}^{K-1} f(2x+1)W_K^{ux}\right) W_K^{u}
\label{fft-oddeven-parens-plusK}
\end{align}


Comparing \cref{fft-oddeven-parens} and \cref{fft-oddeven-parens-plusK}, we see that the expressions within parentheses are identical.
What's more, these parentetical expressions are functionally identical to discrete fourier transforms themselves. Let's notate them as follows:
\begin{align} \label{fft-oddeven-subdfts}
\DFT_u\{f_{\mathrm{even}}(t)\} &:= \sum_{x=0}^{K-1} f(2x)W_K^{ux} \nonumber\\
\DFT_u\{f_{\mathrm{odd}}(t)\} &:= \sum_{x=0}^{K-1} f(2x+1)W_K^{ux}
\end{align}

If we're calculating an $M$ point transform
%\vtodo{vocabulary also how many frequencies do we calculate? same as \# samples? what do we need?}
(i.e. we're wishing to
calculate $F(1), ... , F(M))$, once we've calculated the first $K$ discrete frequencies (i.e. $F(1), \cdots , F(K))$ we may simply reuse the two values we've calculated in \cref{fft-oddeven-subdfts} to calculate the next $F(K+1),..,F(K+K) = F(M)$. Since each expression in parentheses involves $K$ complex multiplications and $K-1$ complex additions, we are effectively saving $K(2K-1)$ calculations in computing the entire spectrum  $F(1), ..,  F(M)$. When $M$ is large, the payoff is undeniable.

In fact, through counting calculations and then doing a proof by induction, we can show that the effective number of calculations is given by $M\log_2{M}$. % see p302 also i have a proof somewhere.

Of course, since \cref{fft-oddeven-subdfts} are DFTs themselves, there's nothing stopping us from reiterating this procedure; if $M$ is substantially large, we can just as easily repeat this process a few times.

Of course, our development was for $1D$.  We can extend this to $2D$ by taking note of \cref{eq:2D-dft-rearrangement}.

%The one caveat is that the above development was for transforming sequences whose lengths are perfect powers of $2$, but we can still

Finally we note the inverse DFT can actually be found via a DFT of the complex conjugate of the original signal, and of course we may translate that operation to a FFT. % see p299.

\section{Calculating the Hessian via FFT: A Demonstration}

% assuming that we already know how the 
Efficient implementation of the Frangi filter ultimately relies on performing a 2D Gaussian blur in frequency space. Here we demonstrate that our FFT implementation of Gaussian blur is commensurate with other implementations. 

In \cref{fig:fft-gaussian-demo}, we demonstrate the compatibility of standard convolution and FFT convolve. Each row corresponds to a different scale at which Gaussian blurring  occurs. Column (a) is standard convolution with a sampled Gaussian kernel, column (b) is FFT-convolution with a Gaussian kernel, and column (c) is a FFT-convolution with the ``discrete Gaussian kernel''. In column (d), the 1D discrete Gaussian kernel (in green) is plotted against the sampled continuous Gaussian kernel (in black). Note that each of the images in the first three columns are scaled the same.
\begin{figure}
  \includegraphics[width=\linewidth]{fft_gaussian_demo}
  \caption{Compatibility of Gaussian convolution strategies}
  \label{fig:fft-gaussian-demo}
\end{figure}

We also demonstrated the ``semigroup property'' of Gaussian kernel convolution.
For a large scale ($\sigma=45$) Gaussian blur ((a) - standard convolution with sampled kernel, (b) FFT with sampled kernel, (c) FFT with discrete kernel), the top row is one round of Gaussian blur with $\sigma=45$ and the bottom row is two progressive passes of Gaussian blur ($\sigma_1 = 10, \sigma_2 = 35$). The mean squared error and mean absolute error between the one-pass and two-pass versions are outputted in \cref{tab:gaussian-mae-mse}. Code for this demo can be found in \texttt{hfft.semigroup\_demo}.
The discrete kernel performs very slightly better than the sampled versions. We originally attempted
this demonstration with a much larger sigma (say $\sigma=150$) and multiple iterations, but unfortunately multiple passes cause the ``noise'' from zeroing out around the boundaries to become very noticable after several iterations. Here, we've opted to crop out a radius of pixels from around the edges equal to the standard deviation of the Gaussian before we calculated the MAE or MSE, to reduce noise from the border.

\begin{table}
  \centering
  \begin{tabular}{c|cc}
    convolution implementation   & MSE & MAE \\
    \hline
    spatial convolution, sampled kernel (A) & 0.00054426 & 0.02015643 \\
    FFT convolution, sampled kernel (B) & 0.00055205 & 0.02029916 \\
    FFT convolution, discrete kernel (C) & 0.00054406 & 0.02015336
  \end{tabular}
\caption{MSE and MAE of One-pass vs. Two-pass Gaussian Blur}
\label{tab:gaussian-mae-mse}
\end{table}

%%REMOVED SEMIGROUP FIGURE, not worth it, just show data

%\begin{figure}
%  \includegraphics[width=\linewidth]{semigroup_demo}
%  \caption{Iterative Gaussian blur}
%  \label{fig:semigroup-demo}
%\end{figure}

We further confirm the commensurate nature of Gaussian blur techniques by comparing the three techniques on a placental image and using each to calculate Frangi targets. The code can be found in \texttt{hfft\_accuracy.py}. In \cref{tab:mse-G-sigma-0.3,tab:mse-F-sigma-0.3,tab:mse-G-sigma-5,tab:mse-F-sigma-5} we compare the mean squared error of a single image blurred (A) with standard spatial convolution, (B) with FFT sampled Gaussian kernel, and (C) with the discrete kernel. We see that the standard convolution and discrete convolution are very similar, while the sampled discrete Gaussian is off by two orders of magnitude, but still reasonably small. We further confirm these by viewing the grayscale intensity of the image from 0 to 1 and the Frangi targets themselves across an arbitrarily chosen horizontal cross section of the image
\cref{fig:cross-sec-demos}, the peaks of the Gaussian blurred image all still occur at the same places, as do the Frangi responses. We repeated this procedure up to $\sigma=90$ and found a situation similar to $\sigma=5$; it was only in very small scales where there was any noticeable difference at all.


\begin{table}
  %\parbox{.45\linewidth}{
    \centering
    \begin{tabular}{c|ccc}
      &  A & B & C \\
      \hline
      A & -  & 1.296e-03 & 6.772e-06 \\
      B & -  & - & 1.247e-03 \\
      C & -  &  - &  - \\
    \end{tabular}
    \caption{MSE of Gaussian Blurs ($\sigma=0.3$)}
    \label{tab:mse-G-sigma-0.3}
  %}
\end{table}
\begin{table}
  %\parbox{.45\linewidth}{
    \centering
    \begin{tabular}{c|ccc}
      &  A &  B         & C \\
      \hline
      A &  - &  4.256e-06 & 5.537e-08 \\
      B &  - &  -         & 4.337e-06 \\
      C &  - &  -         &  - \\
    \end{tabular}
    \caption{MSE of Frangi Scores $\sigma=0.3$}
    \label{tab:mse-F-sigma-0.3}
  %}
\end{table}



\begin{figure}
  \centering
  \includegraphics[width=\linewidth]{Gslice_sigma=3} \\
  \includegraphics[width=\linewidth]{Gslice_sigma=6} \\
  \includegraphics[width=\linewidth]{Fslice_sigma=3} \\
  \includegraphics[width=\linewidth]{Fslice_sigma=6}
  \caption{Image cross-section of Gaussian-blurred (grayscale) placental sample}
  \label{fig:cross-sec-demos}
\end{figure}



\begin{table}
  %\parbox{.45\linewidth}{
    \centering
    \begin{tabular}{c|ccc}
      &  A & B & C \\
      \hline
      A & -  &  9.012e-06 & 8.629e-09 \\
      B & -  & - & 9.031e-06 \\
      C & -  &  - &  - \\
    \end{tabular}
    \caption{MSE of Gaussian Blurs of an Image ($\sigma=5$)}
    \label{tab:mse-G-sigma-5}
  %}
\end{table}
\begin{table}
  %\parbox{.45\linewidth}{
    \centering
    \begin{tabular}{c|ccc}
      &  A &  B         & C \\
      \hline
      A &  - &  9.388e-05 8.383e-07 \\
      B &  - &  -         & 9.599e-05 \\
      C &  - &  -         &  - \\
    \end{tabular}
    \caption{MSE of Frangi Scores $\sigma=5$}
    \label{tab:mse-F-sigma-5}
  %}
\end{table}

Finally, we wish to demonstrate the point of this comparison--that $FFT-based$ convolution is much faster than spatial convolution. We took a much larger sample ($2200 \times 2561$) and timed each method of convolution (average of three trials) for a large number of samples: logarithmic between $\sigma=1$ and $\sigma=128$ with 32 steps. We plot the result in \cref{fig:gconv-runtime}. It shows that the convolution time seems to at least linearly increase with the size of the kernel, whereas FFT is independent of choice of scale. This is to be expected, as convolving with a Gaussian kernel in spatial coordinates requires a greater number of calculations as $\sigma$ increases, whereas the size of the kernel does not change in our frequency space convolution.

\begin{figure}
  \includegraphics[width=\linewidth]{convolution-runtime-demo}
  \caption{Runtime comparison of Gaussian convolution implementations}
  \label{fig:gconv-runtime}
\end{figure}



%\chapter{Morphological Image Processing}
We describe some of the morphological methods in a really basic way and then briefly mention the thinning algorithm.


\section{}
\section{Endpoint detection and compatibility}
We do this in 
\chapter{Research Protocol}
List. All. Decisions. You Make. Be very explicit.

\section{preprocessing}
\section{multiscale setup}
\section{vesselness measure}
Pseudocode?
\section{multiscale merging}
\section{cleanup}

%\chapter{Implementations} \label{ch:implementations}
\vcomment{This chapter shows how things described within the research protocol are performed. By separating it out, I can focus on things like verifying accuracy / comparisons / demos / pseudocode without cluttering up the discussion of the actual methodologies of the next chapter. That way parameters choices, etc. can be more clearly highlighted. However, this section is apt place to discuss how varying parameters influences whatever methods are being used.}


\section{Calculating the Hessian}

% assuming that we already know how the 
Efficient implementation of the Frangi filter ultimately relies on efficient convolution of the image with a gaussian kernel.

Pseudocode for \texttt{np.gradient} which is used in calculating Hessian (code below)
\begin{verbatim}
	gaussian_filtered = fftgauss(image, sigma=sigma)
	Lx, Ly = np.gradient(gaussian_filtered)
	Lxx, Lxy = np.gradient(Lx)
    Lxy, Lyy = np.gradient(Ly)
			\end{verbatim}	

\section Choice of kernel
%% Results and then Analysis and interpretation of the results.

\chapter{Results and Analysis}

use MCC \cite{mcc-original-paper}
\section{Sample visual output}
\subsection{The confusion matrix}
\section{A Source of ``False Negatives'' in the NCS data set} \label{sec:NCS-dataset-issues}

Sometimes the output doesn't agree with the trace, i.e. ``the ground truth'' is not 100\% correct.
sometimes either there's a false negative (reported) but something just wasn't traced in the original  1602443.

\begin{figure} \centering
	\includegraphics[width=0.8\textwidth]{kinds_of_false_positives}
	\caption{''True'' false positives and ``False'' false positives}
\end{figure}

\section{Variations in the Data Set and Imperfections of the Ground Truth}

\begin{enumerate}
\item Collar is stupid and should really be considered like a error in marking the perimeter. Throw these away or edit. Maybe make a section called discarded samples that's stupid but yeah.
\item Vessels suck sometimes. In the portion above, 1602443, there's a random blood clot which gets identified at large $\sigma$. But also the small forked shaped thing which is obviouslly a vessel doesn't get defined.
\item Too much blood (not enough?? no idea) is left in the vessels. leading to the weird white border around some vessels. you could identify these along with black center and combine them somehow. no idea.Also, holy shit, some of the white vessel ``sleeves'' ARE identified in the tracing, and some aren't. Find an example of this and whine about it.
\item Umbillical cord insertion point is stupid and obscures a lot. The tracer guesses but there's no real guiding principle AFAIK..
\item Small vessels aren't accounted for at all. Not sure how to coincide measurement in terms of scale space anymore, but should figure out how to cut off those values before running MCC metric.
\end{enumerate}





\section{Results}

This is a list of particular things I'd like to explore if I have time:

\begin{itemize}
	\item Relationship between traced pixelwidths vs the scale they were pulled from.
	\item Frangi behavior at max scale length and if there's anything that gets too large (related to first derivatives maybe?
	\item calculate the actual weingarten map eigenvectors (although this is
	probably gonna be very fake in a discrete sense).
	\item difference between using green channel and non-green channel.
\end{itemize}

\section{Answer Research Questions}
\chapter{Segmentation} \label{ch:segmentation}

We discuss how we use the Frangi as a prefilter and discuss several different segmentation strategies based upon our Frangi-filtered result. We will compare these methods to an unrelated segmentation strategy, the ISODATA threshold. First, we define some standard quantitative measures of success for segmentation methods.


\section{Binary classifications, scoring methods, and the confusion matrix}

We wish to come up wish a means of gauging the success of an arbitrary segmentation and will adopt a binary classification model to do so.
We end up with a boolean matrix the size of the image, and compare it to the ground truth, the trace, and then compare them to get the number of true positives (TP), true negatives (TN), false positives (FP), and false negatives (FN). We demonstrate these four labels visually with a confusion matrix, as in \cref{fig:sample-confusion}.
% The lighter gray represents true negatives (TN), red represents false negatives (FN), blue represents false positives (FP), and black represents true positives. The darkest gray are is the background mask (pixels are not considered at all in gauging the success of segmentation).
 
\begin{figure}
  \centering
  \includegraphics[height=0.3\textheight]{example_confusion_small}
  \includegraphics[height=0.12\textheight]{confusion_legend}
  \caption{Sample confusion matrix}
  \label{fig:sample-confusion}
\end{figure}

Although there are many measures to gauge the success of binary classification, we will focus on two that we find particularly illustrative. The first is precision, given by

\begin{equation}
\label{eq:precision}
\textrm{precision} \;=\; \frac{TP}{TP+FP}
\end{equation}

and the second is the Matthews Correlation Constant (MCC), given by

\begin{equation} \label{eq:MCC}
MCC = \frac{TP\times TN - FP \times FN}{\sqrt{ (TP + FP)(TP+FN)(TN+FP)(TN+FN)}}
\end{equation}

where precision is a ratio between 0\% and 100\% and the MCC is a measure between -1 and 1. Precision (or positive predictive power) is defined as the ratio between how many pixels were labeled correctly (true positives) and all pixels labeled positive (either correctly or incorrectly). This is a useful score for us--if we are using Frangi as a prefiltering for a more robust technique, then we would rather provide an incomplete starting point, rather than an inaccurate one. Precision therefore does an okay job of representing that scenario: we are not penalized for what we do not label as true, as long as our reports of true are correct.

Of course, we cannot rely on precision as our sole quantitative factor alone--we could simply return everything negative and receive a perfect score of 100\% precision. Therefore we will gauge that measure with that of the MCC \cite{mcc-original-paper}. The main advantage of the MCC is that it is well balanced no matter what the size of the two classes are, and will only be high if the approximation scores well against both labels. A score of $1.0$ means the approximation is 100\% correct, a score of $-1.0$ means that everything is completely incorrect, and a score of $0$ means that the test performs only as well as random guessing. In our analysis, we will consider both the MCC and precision of a particular segmentation simultaneously. We would like an MCC score as high as possible, but will contextually settle for a lower score as long as the approximation is sufficiently precise.

One final point about these measures is that we have decided to report their scores only within the placental plate, rather than the entire rectangular image. Since the area outside of plate is masked from consideration, those points will be true negatives no matter what, and we don't want these points to artificially inflate our score. That being said, we do currently concede one part right now: we will also mask an area around the umbilical cord insertion point, as the large amount of noise here will mean that our scores are artificially low. We would like to remove these areas, but for now we will simply not score them. 

\section{Postprocessing Techniques}
Here we develop four relatively straightforward methods of postprocessing the multiscale Frangi output to obtain an actual PCSVN extraction. Again, we stress that the Frangi filter itself does not produce a segmentation itself. In fact, Frangi in his original paper \cite{frangi-paper} refrained from any explicit analysis of the Frangi score apart from taking the maximum across all scales. Still, we wish to demonstrate some immediate methods of postprocessing these samples in order to illustrate the usefulness of this optimized Frangi filter. 


\subsection{Method A: Fixed Threshold}
Like \cite{huynh2013filter}, we consider a thresholded \Vmax{} as a crude segmentation.
In the fixed threshold method, we choose some threshold $\alpha$ and we say that a pixel $(x,y)$ of the image corresponds to a vessel if
$\Vmax >  \alpha$. This $\alpha$, as noted above, is unfortunately highly dependent on the image domain, and this merging method will tend to happily allow noise generated from scales that are too large or too small if $\Sigma$ is not chosen correctly. We would hope that some normalization of our data set would permit a single choice of $\alpha$ across all samples, but unfortunately we cannot seem to guarantee this. Without prior knowledge of an appropriate choice of $\alpha$, we may have to simply select $\alpha$ by trial and error.

 Another issue with this is the individual scales of the Frangi filter in the extreme cases are not known to scale--although Lindeberg introduced a normalization factor based on the scale to apply to the derivatives, we do not know of an optimal factor to use.

Unfortunately, even with our ``rescaled'' Frangi filter, this $\alpha$ cannot be picked without regard for the particular image domain. Equally problematic, we cannot guarantee that the Frangi filter will decay as our scale exceeds the the bounds where structure is expected to be found.


%\begin{equation}
%{\VSigma}_{\alpha}(x_0,y_0) = \begin{cases}
%1 & \textrm{if}\quad \Vmax(x,y) \; \ge\;  \alpha \\
%0 & \textrm{else}
%\end{cases}  \quad , \; \alpha > 0
%\; \textrm{for}\;\alpha\;\textrm{fixed}.
%\end{equation}

If we insist on such a performing such a thresholding, the ``correct'' choice of $\alpha$ unfortunately seems to depend on the image domain, so user intervention when dealing with the problem domain seems to be the best strategy. A more sincere use of thresholding might be to choose a relatively high threshold, and then use the result for a further technique.
We will discuss alternatives methods of aggregating results from our multiscale method, as well as optimal values for parameters and scales. As a final note, we admit that any future extensions of this work should not hold too much stock in this thresholded result. Analyzing the raw vesselness score \cref{eq:Vmax}, or even the un-merged scale-wise scores \cref{eq:VSigma}, would be far more rewarding.
 
\subsection{Method B: Scalewise Percentile Based Merging}

There is an alternative method of thresholding which avoids the domain-dependent selection of a fixed threshold $\alpha$ altogether. Instead, we could simply select the highest scores from each responses: we calculate a high percentile score and threshold at that value. Due to the large number of zeros outputted by the filter, we opt instead to take the $p$th percentile of only nonzero values of each $\Vsigma$. 
We briefly demonstrate this in \cref{fig:qthresh_demo} on a particularly well-behaved sample. The idea behind percentile-based merging is beneficial for multiscale methods. At each scale, we would like to assume that there is \textit{some} curvilinear content that could be identified. With that in mind, we could simply accept from each scales scores in a very high percentile. We chose for our demonstration a fairly large percentile, $95$, and furthermore bolster this by requiring that any selected pixels be in the 95th percentile of nonzero and unmasked pixels--otherwise the average is artificially low due to the large background and pixels with zero Frangi score. The use of percentiles removes dependence of picking a particular threshold on the problem, while allowing the most prominent features to emerge at each scale

\begin{figure} \centering
	\subfloat[grayscale image]{\includegraphics[width=0.48\linewidth]{qthresh_demo_img}}
	\subfloat[\Vmax]{\includegraphics[width=0.48\linewidth]{qthresh_demo_Fmax}} \\
	\subfloat[scalewise percentile filtering ($p=95$)]{
		\includegraphics[width=0.48\linewidth]{qthresh_demo_q95}}
	\subfloat[scalewise percentile filtering ($p=98$)]{
		\includegraphics[width=0.48\linewidth]{qthresh_demo_q98}} \\
	\caption{Nonzero-percentile thresholding of \Vmax (95th and 98th percentile)}.
	\label{fig:qthresh_demo}
\end{figure}

Of course, the downside of this method is that we do not know in general the size of the network, and  we may miss entire branches of the vascular network if there is a large amount of curvilinear content elsewhere in the image that is simply less prominent than some other features which are prominent at the same scale. An additional caveat of this method is that, since all scales are given equal importance, this method could introduce a large amount of noise if the range of scales probled $\Sigma$, in particular the extremes $\sigma_{\min}$ and $\sigma_{\max}$ are not chosen carefully. That being said, the avantage of this method is that smaller scale features are more reliably extracted, whereas the slightly weaker Frangi signal at smaller scales causes smaller scale phenomena to less likely show up in a fixed threshold extraction.




%\subsection{Method D: Scale Based Sieving}
%
%Our final approach seeks to include not only pixels at each scale that pass a high threshold, but also adjacent pixels at that scale that pass a lower threshold. We proceed as follows. At each scale, take a low threshold, then label each connected region. Then, iterate through each labeled region and add to the final approximation any labeled region that contains a pixel that passes a higher threshold.

\subsection{Method C: Trough Filling}
As we discussed in \cref{sec:signed-frangi-filter}, we can simultaneously calculate the Frangi filter for light and dark curvilinear structures without any added computation time. We shown the signed result of the  Frangi filter at different scales two examples in \cref{fig:signsweep-1} and \cref{fig:signsweep-2}, we can simultaneously calculate calculate the Frangi filter response for bright curvilinear features (dark background) and dark curvilinear features (dark background). These two figures are the analogues to \cref{fig:scalesweep-1} and \cref{fig:scalesweep-2}. Since the Frangi filter normally throws away any response where $\lambda_2 < 0$ (if dark curvilinear features are targeted) or $\lambda_2 >0$ (if light curvilinear features are targeted), we lose no computation time at all by simply keeping both, although we must store more results. After computing the multiscale result, we can easily separate these into a positive and a negative strain, which we will denote $\VSigmapos$ and $\VSigmaneg$, and then merge each as we would with a traditional multiscale Frangi filter, giving us
\Vmaxpos and \Vmaxneg. That is, our \Vmaxpos is the same as our \Vmax in a conventional Frangi filter, and \Vmaxneg is the same result as if we had taken the Frangi filter while only looking for the opposite type (light/dark) curvilinear feature. Plotting $\Vmax$ over a scale of $[-1,1]$ demonstrates an interesting effect. Whereas the Frangi filter generally is not reliable in terms of accurately predicting widths, we \textit{can} get a sense of the width by looking at where there is a relatively strong response of opposite sign.

\begin{figure}[p] \centering
  \subfloat{		\label{fig:signsweep-1p}\includegraphics[width=\linewidth]{{{signsweep_stitch_BN2315363_plate}}}
  } \\[-0.5cm]
  \subfloat{		\label{fig:signsweep-1i}\includegraphics[width=\linewidth]{{{signsweep_stitch_BN2315363_inset}}}
  } \\[-0.5cm]
  \subfloat{		
    \label{fig:signsweep-1c}\includegraphics[width=.75\linewidth]{{{signsweep_colorbar}}}
  } \\
  
  \caption{Signed Frangi output (plate and inset) (Example 1)}
  \label{fig:signsweep-1}
\end{figure}

\begin{figure}[p] \centering
  \subfloat{		\label{fig:signsweep-2p}\includegraphics[width=\linewidth]{{{signsweep_stitch_BN5280796_plate}}}
  } \\[-0.5cm]
  \subfloat{		\label{fig:signsweep-2i}\includegraphics[width=\linewidth]{{{signsweep_stitch_BN5280796_inset}}}
  } \\[-0.5cm]
  \subfloat{		
    \label{fig:signsweep-2c}\includegraphics[width=.75\linewidth]{{{signsweep_colorbar}}}
  } \\
  \caption{Signed Frangi output (plate and inset) (Example 1)}
  \label{fig:signsweep-2}
\end{figure}
That is, at the "foot" of every trough on either side, we can see a bordering curvilinear structure of opposite sign. We perform strict Frangi filtering and separate the positive and negative components. We then perform a different threshold for each signed portion of the Frangi response--a strict one (say $\alphapos > .3$) for the conventional $\Vmax$, and a much looser one for our opposite signed $\Vmaxneg > \alphaneg = .01$. We also truncate the scales we consider for calculating $\Vmax^{(-)}$, considering only the 6 smallest scales of 20, since we empirically notice that the bordering curvilinear features are consistently narrower than the vessels themselves.

We will refer to any pixel where $\Vmaxpos > \alphapos$ as ``in the trough'' (assuming dark curvilinear features), and any pixel where $\Vmaxneg > \alphaneg$ as potentially ``on the lip'' of the trough. The process of ``trough-filling'' then proceeds as follows.  For each pixel within the trough, we iterate over disks of integer radius and dilate the pixel by a disk of that radius if that disk includes a point on the lip, where $\Vmaxneg > \alphaneg$. This allows us to extend the Frangi output from the point where it's strongest all the way to the base of the curvilinear structure, providing a much better indication of the true width of the curvilinear structure.

\begin{figure}[p] \centering
  \subfloat{		\includegraphics[width=\textwidth]{{{fig-insetBN0164923}}}
  } \\[-0.5cm]
  \subfloat{		
    \includegraphics[width=\textwidth]{{{fig-BN0164923}}}
  } \\
  \caption{Trough dilation process (plate and inset)}
  \label{fig:trough_dilation}
\end{figure}

In \cref{fig:trough_dilation} we demonstrate the process of creating a trough dilation segments. The top left is the original sample. This was a $N=20$ Frangi filter with log range from -1.5 to 3.2 and Frangi parameters $\beta=0.15$ and $\gamma=1.0$ The middle top and top right shows $\Vmax^{(+)}$ and $\Vmax^{(-)}$ respectively. The bottom then shows the two markers 
 $\Vmax^{(-)} > \alpha^{(-)} = 0.01$ (in red) and
 $\Vmax^{(+)} > \alpha^{(-)} = .3$ (in blue). We build $\Vmax^{(-)}$ only using the 6 smallest scales, since the lighter curvilinear 'lip' of the trough is of smaller radius than the radius of the trough itself. Probing at too large scales could perhaps introduce noise. We will call this subset $\Sigmaneg \subset \Sigma $ and choose $\Sigmaneg=\{\sigma_1 , ... , \sigma_6\}$.
 
 

\section{Nonfrangi Segmentation Methods}

Of course, there are many other viable methods of segmentation. We refer to \cite{anghel2018placental} for several high-performing (albeit considerably resource intensive) segmentation methods on this dataset, namely involving shearlets, Laplacian eigenmaps, and a conditional generative adversarial network.

We will be content in the present paper to simply compare our results to a relatively fast technique that is not based on differential geometry or morphology, but instead that of a global threshold on the grayscale image. Although many viable thresholding algorithms exist, we opt for an intermeans threshold, or the Ridler-Calvert or ISODATA method, as described in \cite{isodata}, which is implemented in \texttt{scikit-image} as \texttt{filters.threshold\_isodata}. This functions uses an iterative process to find the smallest threshold which is midway between the mean intensities of the high pixels and the low pixels. In other words, the optimal $\alpha_{\textrm{ISO}}$ satisfies

\begin{equation} \label{eq:ISODATA}
\alpha_{\textrm{ISO}} = \arg\min_{\alpha} \left( \frac{1}{2} \Big[
\textrm{mean}\left\{ \img(x,y) \;|\; \img(x,y) \le \alpha \right\} 
+
\textrm{mean}\left\{ \img(x,y) \;|\; \img(x,y) >   \alpha \right\}
 \Big]
 \right)
\end{equation}

Since the vascular structure in our image domain is darker than the background, we select pixels
where $\img(x,y) < \alphaiso$.



\section{Comparison of Segmentation Methods}
In our demonstration of segmentation methods across all 201 samples in our image domain, we use the preprocessing procedure described in \cref{ch:research-protocol}.

Our multiscale Frangi filter setup involves 20 scales spaced logarithmically from $\sigma_1 = 2^{-1.5} \approx .35$ to $\sigma_{20} = 2^{3.2} \approx 9.20$. We will compare the results of 6 different segmentation methods using three different parametrizations of the Frangi filter. These parametrizations and segmentation methods with specific parameter choices are summarized in \cref{tab:segmentation-methods,tab:segmentation-parametrizations}.


\begin{table}[h]
\centering
\begin{tabular}{|c|l|p{4.5cm}|}
  \hline
  Label & Description & Parameter(s) \\ \hline
  thresh-high & Fixed threshold of Frangi filter $\Vmax > \alpha $ & $\alpha = 0.3$ \\ \hline
  thresh-low &  Fixed threshold of Frangi filter $\Vmax > \alpha $ & $\alpha = 0.2$ \\ \hline
  snz-p-high & scalewise nonzero percentile filtering of \VSigma & $q = 95$ \\ \hline
  snz-p-low & scalewise nonzero percentile filtering of \VSigma & $q = 98$ \\ \hline
  TF & trough-filling method &
    \parbox{4.5cm}{$\alphapos=0.3,\;\alphaneg=.01,\\
                    \Sigmaneg=\{\sigma_1,\cdots,\sigma_6\}$} \\ \hline
  ISODATA & Non-Frangi global threshold & see \cref{eq:ISODATA} \\ \hline
\end{tabular}
\caption{Summary of Segmentation Methods}
\label{tab:segmentation-methods}
\end{table}

\begin{table}[h]
  \centering
  \begin{tabular}{|c|c|c|}
    \hline
    Label  & $\beta$ & $\gamma$ \\ \hline
    standard & $0.5$ & $0.5$ \\ \hline
    semistrict & $0.15$ & $0.5$ \\ \hline
    strict & $0.15$ & $1.0$ \\ \hline
  \end{tabular}
\caption{Summary of Frangi Parametrizations for Segmentation Demo} 
\label{tab:segmentation-parametrizations}
\end{table}


In \cref{fig:seg-montage-example}, we look at the results of our various segmentation procedures for a typical well behaved sample. We can see that there are fewer false negatives for strict parametrization than with a standard parametrization. Even though the best MCC score across the 10 demonstrated in \cref{fig:seg-montage-example} is achieved by trough filling under standard parametrization, we note that trough filling in the strict parametrization has comparatively fewer false negatives and much higher precision. The highest precision is achieved by our higher fixed threshold. Thus, we should see the high MCC score achieved by the trough filling method as a testament to the usability of a simpler threshold (such as FT-high) as a precursor to more complete segmentation methods.


In \cref{fig:scoring-boxplots} we graph the MCC and precision scores across our entire set of 201 images. In each boxplot, the median is labeled, with first and third quartiles making up the edges of the box. The ends of the whiskers represent  $median \pm 1.5*(Q3-Q1)$. The quantity ($Q3-Q1$), the difference of the third quartile (where $p=75$) and first quartile (where $p=25$) is the so-called interquartile range \cite{scipy}. Outliers outside of those whiskers, from samples such as those in \cref{fig:bad-gallery}, are plotted as individual data points.

Across all samples, maximum precision was achieved by the higher fixed threshold of $\alpha =0.3 $ in all but 22 samples, and the trough filling method offered the best MCC in about three-fourths of the samples (51 of 201). Where the trough filling method was not most accurate, we noticed that the $q=95 $ nz-percentile method had a slightly higher MCC score. From viewing the samples, we see that noise from the umbillical cord insertion point still hinders a number of samples, causing noise within the Frangi filter. 

From viewing \cref{fig:scoring-boxplots} we can see that, for any of the
three parametrizations, we achieve a higher precision and lower MCC from increasing the fixed threshold. The same trend occurs between scalewise nz-p segmentations when we increase the quartile (from q=95 to q=98). The median MCC of the ISODATA method was lower than all Frangi-based segmentations across all with the exception of the highest fixed threshold under strict parametrization--although the latter method has the highest median precision score by far. ISODATA has a much lower precision than any of the other methods present here.

We notice a larger number of MCC outliers for scalewise methods than fixed threshold based methods. We theorize that this is due to the difficulty in knowing \textit{a apriori} an appropriate range of scales to probe. Future work therefore could immediately improve upon the results of this chapter by finding an appropriate range or, more likely, a normalization factor for scales that makes the success of segmenation less dependent on $\sigma_{min}$ and $\sigma_{max}$.

From \cref{fig:scoring-boxplots} we also notice that the MCC and precision for each scalewise method is not as affected by changes in parameterization as the fixed methods. We suggest that is because stricter parameters have the effect of rescaling to some degree within each step of the multiscale method, so that the same pixels at each scale ultimately occur at each scale's highest percentiles.

Looking at \cref{fig:seg-montage-example2}, we see a situation in which scalewise methods do poorly. The $p=95$ scalewise method under either parametrization, as well as the lower fixed threshold under standard parametrization show some curvilinear noise between vessels. These come from larger scales, so there is apparently no signficant vasculature being picked up at the largest scales, causing noise to appear. This is less pronounced for fixed thresholding using stricter parameters, although some of this noise can still be seen in the strict Frangi's \Vmax plot.


\begin{figure}[p] \centering
  \subfloat[standard parametrization]{ 	\includegraphics[width=\textwidth]{fig-BN2481864-standard-seg}} \\
  \subfloat[strict parametrization]{ 	\includegraphics[width=\textwidth]{fig-BN2481864-strict-seg} }
  \caption{Segmentation results, example 1 (standard and strict parametrization)}
	\label{fig:seg-montage-example}
\end{figure}

\begin{figure}[p] \centering
	\subfloat[standard parametrization]{ 	\includegraphics[width=\textwidth]{fig-BN7753462-standard-seg}} \\[-0.5cm]
%    \subfloat{ 	\includegraphics[width=\textwidth]{fig-BN7753462-semistrict-seg} } \\[-.5cm]
	\subfloat[strict parametrization]{	\includegraphics[width=\textwidth]{fig-BN7753462-strict-seg} }
	\caption{Segmentation results, example 2 (standard and strict parametrization)}
	\label{fig:seg-montage-example2}
\end{figure}

\begin{figure}[p] \centering
  \subfloat[standard Frangi parametrization ($\beta=0.5, \gamma=0.5$)]{  
            \includegraphics[width=0.5\textwidth]{all-MCC-boxplot-standard} 
            \includegraphics[width=0.5\textwidth]{all-precision-boxplot-standard} } \\
  \subfloat[semistrict Frangi parametrization ($\beta=0.15, \gamma=0.5$)]{ 
            \includegraphics[width=0.5\textwidth]{all-MCC-boxplot-semistrict} 
            \includegraphics[width=0.5\textwidth]{all-precision-boxplot-semistrict} } \\
  \subfloat[strict Frangi parametrization ($\beta=0.15, \gamma=1.0$)]{ 
            \includegraphics[width=0.5\textwidth]{all-MCC-boxplot-strict} 
            \includegraphics[width=0.5\textwidth]{all-precision-boxplot-strict} } \\
      \caption{MCC and precision of segmentation methods (201 samples)}
      \label{fig:scoring-boxplots}
\end{figure}

%\section{Scale-wise Random Walker: A demo}
%
%
%We observed that areas where Frangi scores are zero in well-behaved samples seem to neatly outline prominent vascular features. Following this idea, we employed a random walker segmentation \cite{Grady-Random-Walks} (which is implemented by \texttt{scikit-image}). Random walk segmentation comes about by solving a diffusion problem over a discrete array (in this case, the Frangi vesselness score itself) given starting markers. At each scale, we positively labeled pixels whose Frangi score was very high ($\Vsigma(x_0,y_0) > .4$), and negatively labeled pixels whose score was $0$ (i.e. where the leading principal eigenvalue was positive). The result of this technique is demonstrated in \cref{fig:rw-demo-scalewise} and the result (along with the original sample for comparison) is shown in \cref{fig:rw-demo-merged}.
%In \cref{fig:rw-demo-scalewise}, the first column is the Frangi vesselness score at that scale, where black is a score of 0, to emphasize the difference between a score of zero and even a very small positive score, which appear in blue. The middle score are markers passed to the random walker--blue are seeds labelled with a ``1'' (where the Frangi vesselness score is 0), green is labeled ``2'' (where \Vmax > .4), and purple represents unknowns that will either assigned either label. In the last column, the result of the random walker is given--areas that have been added to the label ``2'' are shown in yellow. Although the result of random walker segmentation is technically a binary matrix, we still show the original seeds of label 2 in green for easier comparison. Similarly, the purple in the right column has actually been labeled ``1'' for non-vascular, but is left in its original color to emphasize what was assigned background. In \cref{fig:rw-demo-merged} we show the original image and the result of merging all positively marked pixels at each scale. Black means the pixel was unmatched, while increasing colors of blue (larger scales) to white (smaller scales) indicate the smallest scale from which a pixel was marked after the random walker technique. \vtodo{Do I need to over this in math methods section?} Though we shall set up the multiscale method slightly differently in \cref{ch:results-analysis}, we used a Frangi anisotropy coefficient of $\beta=0.35$ , and 12 scales logarithmically spaced from $\sigma_1 = 2^{-1.5} $ to $\sigma_{12} = 2^{3.5}$ to generate these figures. There is a parameter for the random walker algorithm (unfortunately also called $\beta$) which serves as a diffusion penalization coefficient (larger values making diffusion over the image less likely). We used \texttt{scikit-image}'s default value of 130. 
%
%\begin{figure}[p] \centering
%  \subfloat{
%    \includegraphics[width=\textwidth]{rw_demo_scale_03}
%  }\\[-0.5cm]
%  \subfloat{
%    \includegraphics[width=\textwidth]{rw_demo_scale_05}
%  }\\[-0.5cm]
%  \subfloat{
%    \includegraphics[width=\textwidth]{rw_demo_scale_07}
%  }\\[-0.5cm]
%  \subfloat{
%    \includegraphics[width=\textwidth]{rw_demo_scale_10}
%  }
%  \caption{Scale-wise random walker segmentation (select scales)}
%  \label{fig:rw-demo-scalewise}
%\end{figure}
%
%
%\begin{figure}[t] \centering
%  \subfloat{
%    \includegraphics[height=0.4\textwidth]{rw_demo_base}
%  }
%  \subfloat{
%    \includegraphics[height=0.4\textwidth]{rw_demo_labels}
%  }
%  \caption{Random walker segmentation (sample and merged result)}
%  \label{fig:rw-demo-merged}
%\end{figure}
\section{Network Completion} \label{sec:network-completion}
We can see from looking at the \Vmax outputs of our samples that even where gaps exist in the segmentation output, there is still \textit{some} Frangi output where the gap exists. For various reasons, the output might not be as strong at that particular region, but we can use that fact to bridge some of these gaps. To do so would be to achieve ``simple gap completion''--that is, connecting the vascular network to compensate for the shortcomings of a particular segmentation method. We differentiate this from ``predictive gap completion,'' which we will use to signify the problem of connecting the vascular network where the tracer themselves had to guess what was happening with the network. For example, the entire area surrounding the umbilical cord insertion point in \cref{fig:seg-montage-example2} is difficult to make out. The tracer ultimately used some knowledge of vascular network growth to predict what was happening in this region to produce the ground truth. Predictive gap completion would also need to be used to solve crossings of veins and arteries.

First, once we've produced a segmentation, we use morphological thinning to reduce it to one pixel width with \cite{thinning} and look for endpoints of an otherwise connected point. Using a $3\times 3$ structuring element, we iterate over each pixel and identify how many local neighbors it has. If a pixel has zero or one local neigbors, we identify it as an endpoint of the partial network. After identifying these endpoints, we assign each a label $(i,j)$ depending on where the neighboring pixel is located, as according to \cref{fig:endpoint_labels}. We deem two endpoints potentially connectible only if they're not connected on the same side. That is, their labels $(i,j)$ and $(i',j')$ must have $i\ne i'$ or $j\ne j'$ (unless $i$ or $j$ is 1). For example, if an endpoint is connected to the partial network on its top side, any endpoint that connects to it cannot also be connected to a network on its top. If a pixel has no connections at all, with label $(1,1)$, we do not restrict its connetions at all. To save time (though we don't anticipate it will affect the result much), we also restrict pais from being more than a set distance away (in this case, 100 pixels in Euclidean length).

\begin{figure}
	\centering
	\includegraphics[width=.5\textwidth]{endpoint_labels}
	\caption{Endpoints labels based on adjacent neighbor location}
	\label{fig:endpoint_labels}
\end{figure}

After we limit the connections, we consider each pair of endpoints and draw a straight line segment between the two. If that passes through a point where $\Vmax$ is 0, we disallow that pair as well. We also disallow any line which crosses any part of the network which is known to exist.
\begin{figure}[p]
	\includegraphics[height=0.4\textheight]{paths_between_endpoints_positive_score}
	\caption{All lines between endpoints with nonzero $\Vmax$}
	\label{fig:network-completion-connected-pairs}
\end{figure}
\begin{figure}[p] \centering
	\includegraphics[height=0.4\textheight]{completed_by_nearest_to_line_mash}
	\caption{Partially completed network}
	\label{fig:network-completion-end-result}
\end{figure}

Finally, from the list of all remaining pairs of endpoints, we simply select the path along which the maximum mean value of $\Vmax$ is achieved. \cref{fig:network-completion-connected-pairs} shows non-violating paths between end-point pairs. The coloration shows the average value of $\Vmax$ along any straight path between compatible endpoints for which there does not occur any pixel with zero $\Vmax$. From these we can choose the path with largest average \Vmax. This partially completed network is shown in  \cref{fig:network-completion-end-result}(in yellow) overlaid on $\Vmax$.
We show this result as an indication of what can be done with the simple generation of $\Vmax$, although we simply demonstrate it here rather than including it in our main segmentation results. This is because, for its complexity, we hope we can instead solve both simple and predictive gap completion using a single method.

\begin{figure}[t] \centering
	\includegraphics[width=0.8\linewidth]{pd_both}
	\caption{Approximated principal directions at endpoints}
	\label{fig:pd_demo}
\end{figure}

Finally, one final comment we can make. For the scale the maximum Frangi output occurs, at, we can look at the Hessian matrix itself to glean additional information. Recalling our discussion from \cref{ch:diffgeo}, we can instead of calculating the leading eigenvalue, we can calculate the leading \textit{eigenvector.} Just as the leading eigenvalue of the Hessian is a good approximation of the principal curvature at that point, the leading eigenvector is a good approximation of the principal direction. In \cref{fig:pd_demo} we show the vectors of the leading (and trailing) eigenvectors of each endpoint of the thinned, approximated network. We note that these could additionally be used towards the goal of network completion, but again we hesitate and instead look for a better overarching method that would address the gap problem.


\section{Further Research Directions}

Future research will first aim toward continue automating more of the preprocessing, specifically toward an even higher success rate of identifying the placental perimeter, umbilical cord insertion point, and any cuts without relying on a manual trace.  As mentioned in \cref{ch:research-protocol}, a more careful preparation of samples (i.e. as the picture is taken) would alleviate some of the difficulty of image registration. We also should improve our glare reduction algorithm, as it currently relies on an arbitrary threshold. We also need to remove the umbilical cord stump, as a large amount of noise around that point is still causing a large amount of noise in many samples. As far as our multiscale method goes, we would like to develop a notion of automatic scale selection, that would help us better normalize smaller scales, as well as allow us to find a specific largest scale (as some vessels in specific samples were only easily identified at very large scales, whereas these scales would introduce only noise in most other samples).
Finally, we strongly suggest further investigation of the usefulness of the Weingarten-based Frangi filter as compared to the convential Hessian-based one, as we briefly discussed in \cref{sec:wein-frangi}.

Any additional research on this problem that wishes to use the Frangi filter as a prefilter  (e.g. the trough completion method detailed in \cref{ch:segmentation} or some more involved algorithm) will likely need to solve the network completion problem, which we addressed briefly in \cref{sec:network-completion}.

%\chapter{Primitive Network Completion}

In this chapter we discuss the idea of network completion and demonstrate a relatively brute-force method of using a segmentation result to solve some of the network completion problem.

One of the biggest shortcomings of our demonstrated segmentation methods is that they frequently result in gaps in parts of the network. We demonstrate a way to rectify this problem in certain situations, and then discuss how to extend these arguments to fill larger, more uncertain gaps.

First, once we've produced a segmentation, we thin it down with \cite{thinning} and look for endpoints of an otherwise connected point. Using a $3\times 3$ structuring element, we iterate over each pixel and identify how many local neighbors it has. If a pixel has zero or one local neigbors, we identify it as an endpoint of the partial network. After identifying these endpoints, we assign each a label $(i,j)$ depending on where the neighboring pixel is located, as according to \cref{fig:endpoint_labels}. We deem two endpoints potentially connectible only if they're not connected on the same side. That is, their labels $(i,j)$ and $(i',j')$ must have $i\ne i'$ or $j\ne j'$ (unless $i$ or $j$ is 1). For example, if an endpoint is connected to the partial network on its top side, any endpoint that connects to it cannot also be connected to a network on its top. If a pixel has no connections at all, with label $(1,1)$, we do not restrict its connetions at all. To save time (though we don't anticipate it will affect the result much), we also restrict pais from being more than a set distance away (in this case, 100 pixels in Euclidean length).

\begin{figure}
	\includegraphics[width=.85\textwidth]{endpoint_labels}
	\caption{Endpoints labels based on adjacent neighbor location}
	\label{fig:endpoint_labels}
\end{figure}

After we limit the connections, we consider each pair of endpoints and draw a straight line segment between the two. If that passes through a point where $\Vmax$ is 0, we disallow that pair as well. We also disallow any line which crosses any part of the network which is known to exist.
Finally, from the list of all remaining pairs of endpoints, we simply select the path along which the maximum mean value of $\Vmax$ is achieved. \cref{fig:network-completion-all-pairs} shows non-violating paths between end-point pairs, and \cref{fig:network-completion} shows a partially completed network (in yellow) overlaid on $\Vmax$.



\begin{figure}
	\includegraphics[width=\linewidth]{paths_between_endpoints_positive_score}
	\caption{All lines between endpoints with nonzero $\Vmax$}
	\label{fig:network-completion-all-pairs}
\end{figure}
\begin{figure}
	\includegraphics[width=\linewidth]{completed_by_nearest_to_line_mash}
	\caption{Partially completed network}
	\label{fig:network-completion-end-result}
\end{figure}

We show the complete process for comparison on two different samples. 

\begin{figure}
	\includegraphics[width=\linewidth]{sample_network_completion}
	\caption{Trough Dilation and Network Completion (Example 1)}
	\label{fig:network-completion-demo-1}
\end{figure}
\begin{figure}
	\includegraphics[width=\linewidth]{sample_network_completion_2}
	\caption{Trough Dilation and Network Completion (Example 2)}
	\label{fig:network-completion-demo-2}
\end{figure}



\chapter{Conclusion} \label{ch:conclusion}

We justified the use of differential geometry in 2D discrete image processing, and vastly improved upon the implementation of the Frangi filter. Our improved implementation allowed us to take more steps in our multiscale method and thus choose stricter parameters for Frangi scale. We used our multiscale Frangi vesselness measure to suggest several alternative approaches at merging the vesselness and compared their effectiveness as a precursor to segmentation and eventually network completion.

\section{Future research directions} \label{sec:future-research-directions}


Our goal is to eventually solve the network connection problem.
\begin{itemize}

	\item Solve the Network Connection Problem (PICTURE OF GAPS)
	Try something like \cite{laptev2000automatic} or use of principal curvatures.
	\item Implement the automatic scale selection and normalization of derivates as mentioned in Lindeberg \cite{lindeberg1998feature} to relieve ourselves of
        our current dependency on manual selection of $\sigma_{\min}$ and  $\sigma_{\max}$.
	\item Look into gradient prefiltering more as well as varying $\gamma$ more, especially in areas where we suspect the network could be completed.
  \item Look into using signed frangi arguments.
	\item Use this as preprocessing for a Neural Network.
	(cite kara's work, katalinas work)
	\item Apply to more image domains (STARE, other placental domains).
	\end{itemize}
\begin{figure}[p] \centering
	\subfloat{		\label{fig:signsweep-1p}\includegraphics[width=\linewidth]{{{signsweep_stitch_BN2315363_plate}}}
	} \\[-0.5cm]
	\subfloat{		\label{fig:signsweep-1i}\includegraphics[width=\linewidth]{{{signsweep_stitch_BN2315363_inset}}}
	} \\[-0.5cm]
	\subfloat{		
		\label{fig:signsweep-1c}\includegraphics[width=.75\linewidth]{{{signsweep_colorbar}}}
	} \\
	% should i use the real sample names? or obfuscate?
	\caption{Signed Frangi output (plate and inset) (Example 1)}
	\label{fig:signsweep-1}
\end{figure}

\begin{figure}[p] \centering
	\subfloat{		\label{fig:signsweep-2p}\includegraphics[width=\linewidth]{{{signsweep_stitch_BN5280796_plate}}}
	} \\[-0.5cm]
	\subfloat{		\label{fig:signsweep-2i}\includegraphics[width=\linewidth]{{{signsweep_stitch_BN5280796_inset}}}
	} \\[-0.5cm]
	\subfloat{		
		\label{fig:signsweep-2c}\includegraphics[width=.75\linewidth]{{{signsweep_colorbar}}}
	} \\
	% should i use the real sample names? or obfuscate?
	\caption{Signed Frangi output (plate and inset) (Example 1)}
	\label{fig:signsweep-2}
\end{figure}
f


\HalfPage{Appendices}
%% Renews chapter command so type appendix files in the same way


\myappendix{Code Listings}\label[appendix]{app:listings}
%% Here is the appendix (not needed.)

Put code here (and on github)
\myappendix{3D Visualization of the Frangi Filter}\label[appendix]{app:extra-figures}

\begin{figure}[p]
  \centering
\includegraphics[width=\linewidth]{frangi3dpart0}
\caption{3D graph of the Frangi Vesselness Measure, variable $\gamma,\; \beta=0.1$}
\end{figure}

\begin{figure}[p]
  \centering
  \includegraphics[width=\linewidth]{frangi3dpart1}
  \caption{3D graph of the Frangi Vesselness Measure, variable $\gamma,\; \beta=0.25$}
\end{figure}

\begin{figure}[p]
  \centering
  \includegraphics[width=\linewidth]{frangi3dpart2}
  \caption{3D graph of the Frangi Vesselness Measure, variable $\gamma,\; \beta=0.5$}
\end{figure}

\begin{figure}[p]
  \centering
  \includegraphics[width=\linewidth]{frangi3dpart3}
  \caption{3D graph of the Frangi Vesselness Measure, variable $\gamma,\; \beta=0.9$}
\end{figure}

\begin{figure}[p]
  \centering
  \includegraphics[width=\linewidth]{frangi3dpart4}
  \caption{3D graph of the Frangi Vesselness Measure, variable $\gamma,\; \beta=1$}
\end{figure}

\begin{figure}[p]
  \centering
  \includegraphics[width=\linewidth]{frangi3dpart5}
  \caption{3D graph of the Frangi Vesselness Measure, variable $\gamma,\; \beta=1.5$}
\end{figure}



%Here is the list of cited works.
\HalfPage{Bibliography}

%% then export in BibTeX format into file thesis_bib.bib
\bibliographystyle{ieeetr}
\bibliography{thesis_bib}

\end{document}
