 \chapter{The Frangi Filter: A multiscale approach} \label{sec:frangi-multiscale}
    
     With the ideas of scale established, we may return to our discussion of the Frangi filter.
    Our ideas of scale developed in the previous section imply that, if the ridgelike structures we wish to detect are more prominent at different scales, then a multiscale approach is the natural one. Considering our
    developments in \cref{sec:frangi}, we wish to probe at multiple scales
    regions that would receive a high vesselness score at any range,
    and consider them all together. Frangi \cite{frangi-paper} approached this problem by simply aggregating vesselness measure over all scales:
    
    \begin{equation} \label{frangi-vesselness-max}
    V(x_0, y_0) = \underset{\sigma \in \Sigma}{\max}\;  V_\sigma(x_0, y_0)
    \end{equation}
    
    where $\Sigma := \left\{ \sigma_0, \sigma_1 , \cdots, \sigma_N \right\}$ is
    a range of parameters at which to probe. These should be chosen to be representative enough of all scales where meaningful content is expected to be found.
    
    % move to Chapter 3.
    \subsection{Thresholding}
    
    After this procedure, we are left with a matrix with as many samples/pixels as the original image, all with a vesselness measure between $0$ and $1$ for each pixel in the image:
    
    \begin{equation} \label{eq:frangi-max-matrix}
    \mathsf{V}_\Sigma := \left[ V(x, y)\right]_{\substack{0\le x<M \ 0\le y<N}}
    \end{equation}
    
    Notably, Frangi \cite{frangi-paper} refrained from explicitly interpreting the probablility assigned by \cref{frangi-vesselness-max}; that is--whether a particular point $(x,y)$ in the image definitely a vessel or not. Instead, he cautioned that the result should not be used as a segmentation method alone, and that the size
    of the vasculature cannot be determined rigorously from the filter alone.
    
    However, for the purposes of obtaining an intermediate result, we wish to be final about the whole matter and ultimately say whether or not a pixel does in fact corresponds to a curvilinear structure. A straightforward enough approach is to simply threshold at some fixed value. The resulting matrix can be given in terms of either \cref{frangi-vesselness-max} or \cref{eq:frangi-max-matrix}
    \vcleanup{fix notation in all of this}
    \begin{equation}
    V_{\Sigma,\alpha}(x,y) = \begin{cases}
    1 & \textrm{if}\quad V(x,y) \; \ge\;  \alpha \\
    0 & \textrm{else}
    \end{cases}  \quad , \quad \alpha > 0
	    \; \textrm{for } \; \alpha \;\textrm{fixed}.
    \end{equation}
    
	We will discuss alternatives methods of aggregating results from our multiscale method, as well as optimal values for parameters and scales
	in \cref{ch:implementations}. As a final note, we admit that any future extensions of this work (as will be discussed in \cref{ch:conclusion}) should not hold too much stock in this thresholded result, and analyzing the 
	raw vesselness score \cref{eq:frangi-max-matrix}, or even
	the un-merged scale-wise scores, would be far more
	rewarding.    
	

All that remains to describe mathematically is how to actually calculate the derivatives of our images and deal with the ultimately discrete nature of our samples.    