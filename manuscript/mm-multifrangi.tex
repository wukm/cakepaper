 \chapter{The Frangi Filter: A multiscale approach} \label{sec:frangi-multiscale}
    
     With the ideas of scale established, we may return to our discussion of the Frangi filter.
    Our ideas of scale developed in the previous section imply that, if the ridgelike structures we wish to detect are more prominent at different scales, then a multiscale approach is the natural one. Considering our
    developments in \cref{sec:frangi}, we wish to probe at multiple scales
    regions that would receive a high vesselness score at any range,
    and consider them all together. Frangi \cite{frangi-paper} approached this problem by simply taking the maximum vesselness measure over all scales. Thus the multiscale Frangi vesselness score at the pixel $(x_0, y_0$) would be 
    
    \begin{equation} \label{eq:Vmax}
    \Vmax(x_0, y_0) =
    	\underset{\sigma \in \Sigma}{\max}\left\{  \Vsigma (x_0, y_0) \right\}
    \end{equation}
    
    where $\Sigma := \left\{ \sigma_0, \sigma_1 , \cdots, \sigma_N \right\}$ is
    the range of scales at which to probe. These should be chosen to be representative enough of all scales where meaningful content is expected to be found.
    
   
    \subsection{Thresholding}
    
    After the maximization in \cref{eq:Vmax}, we are left with a matrix with as many pixels as the original image, all with a vesselness measure between $0$ and $1$ for each pixel in the image.
       
    At this point, Frangi \cite{frangi-paper} refrained from explicitly interpreting the score assigned by \cref{eq:Vmax}; that is--whether a particular point $(x,y)$ in the image definitely a vessel or not. Instead, he cautioned that the result should not be used as a segmentation method alone, and moreover that the width of the vasculature cannot be determined rigorously from the filter alone.   
   

All that remains to describe mathematically is how to actually calculate the derivatives of our images and deal with the ultimately discrete nature of our samples.    