\chapter{Implementations} \label{ch:implementations}
\vcomment{This chapter shows how things described within the research protocol are performed. By separating it out, I can focus on things like verifying accuracy / comparisons / demos / pseudocode without cluttering up the discussion of the actual methodologies of the next chapter. That way parameters choices, etc. can be more clearly highlighted. However, this section is apt place to discuss how varying parameters influences whatever methods are being used.}


\section{Calculating the Hessian}

% assuming that we already know how the 
Efficient implementation of the Frangi filter ultimately relies on efficient convolution of the image with a gaussian kernel.

Pseudocode for \texttt{np.gradient} which is used in calculating Hessian (code below)
\begin{verbatim}
	gaussian_filtered = fftgauss(image, sigma=sigma)
	Lx, Ly = np.gradient(gaussian_filtered)
	Lxx, Lxy = np.gradient(Lx)
    Lxy, Lyy = np.gradient(Ly)
			\end{verbatim}	

\section Choice of kernel