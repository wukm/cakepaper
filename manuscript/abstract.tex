%%%%% Abstract

% University Guidelines suggest less than 150 words or 2 pages

Recent statistical analysis of placental features has suggested the usefulness
of studying key features of the placental chorionic surface vascular network
(PCSVN) as a measure of overall neonatal health \cite{chang2017}. A recent
study has suggested that reliable reporting of these features may be useful in
identifying risks of certain neurodevelopmental disorders at birth. The
necessary features can be extracted from an accurate tracing of the surface
vascular network, but such tracings must still be done manually, with
significant user intervention. Automating this procedure would not only allow
more data acquisition to study the potential effects of placental development on
later conditions, but even perhaps provide a real-time diagnostic for neonatal
risk factors.

Much work has been to develop reliable vascular network segmentation methods for
well-known image domains (such as retinal MRA images) using Hessian-based
filters, namely the multiscale Frangi filter. It is desirable to extend these
techniques to study placental images, but the approach has been historically
hindered by the comparative irregularity of the placental surface as a whole,
which introduces significant noise into filtered result.  Previous work has either
involved additional filtering \cite{huynh2013filter} or other techniques that are
often time-consuming and resource intensive \cite{djima2017enhancing}.

Here we provide an in-depth mathematical background of the multiscale Frangi filter.
Informed by this theory, we are able to identify stricter parameters that allow us
to greatly improve our result. We also reimplement the Frangi filter in frequency space
(using a fast fourier transform), which allows us to quickly probe many scales.

We demonstrate the effectiveness of our sped-up implementation of the Frangi
filter by performing a large (N=20) multiscale Frangi filter on a set of 201
placental images from a private database provided by the National Children's
Study (NCS). We then compare several approaches of merging the multiscale
result into an approximation of the PCSVN and compare them to manual tracings
of the network. Finally, we develop the notion of the \textit{signed} Frangi filter,
upon which we describe a novel-yet-straightforward segmentation method called "trough filling".


