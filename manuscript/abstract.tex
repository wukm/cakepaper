%%%%% Abstract

% University Guidelines suggest less than 150 words or 2 pages

Recent statistical analysis of placental features has suggested the usefulness
of studying key features of the placental chorionic surface vascular network
(PCSVN) as a measure of overall neonatal health \cite{chang2017}. A recent
study has suggested that reliable reporting of these features may be useful in
identifying risks of certain neurodevelopmental disorders at birth. The
necessary features can be extracted from an accurate tracing of the surface
vascular network, but such tracings must still be done manually, with
significant user intervention. Automating this procedure would not only allow
more data acquisition to study the potential effects of placental health on
later conditions, but may ideally serve as a real-time diagnostic for neonatal
risk factors as well.

Much work has been to develop reliable vascular extraction methods for
well-known image domains (such as retinal MRA images) using Hessian-based
filters, namely the (multiscale) Frangi filter. It is desirable to extend these
technique to study placental images, but this approach is greatly hindered by
the comparative irregularity of the placental surface as a whole, which
introduces significant noise into the image domain.  Prior work
\cite{huynh2013filter} has made to apply an additional local curvilinear filter
to the Frangi result in an effort to remove some noise from the final
extraction.

Here we provide an in depth mathematical background of the Frangi filter and a
reasonable introduction to Gaussian scale space theory. Finally, we discuss an
important advancement in implementation--scale space conversion for
differentiation (i.e. gaussian blur) via Fast Fourier Transform, which offers a
significant speedup. This allows us faster calculation of the eigenvalues of
the Hessian, from which we calculate the Frangi filter, a vesselness measure.

We demonstrate the effectiveness of our sped-up implementation of the Frangi
filter by performing a large (N=40) multiscale Frangi filter on a set of 201
placental images from a private database provided by the National Children's
Study (NCS). We then compare several approaches of merging the multiscale
result into an approximation of the PCSVN and compare them to manual tracings
of the network. We finally suggest several ways to improve upon our
approximation, namely by using the Frangi result as a prefilter for more robust
techniques, providing a brief demo using a random walker segmentation.


