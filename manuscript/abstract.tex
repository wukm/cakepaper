%%%%% Abstract

% University Guidelines suggest less than 150 words or 2 pages
%%%%% NOTE: This abstract dates from September 28, 2017 and may not reflect the current paper. Please alter as necesary.

Recent statistical analysis of placental features has suggested the usefulness of studying key features of the placental chorionic surface vascular network (PCSVN) as a measure of overall neonatal health. A recent study has suggested that reliable reporting of these features may be useful in identifying risks of certain neurodevelopmental disorders at birth. The necessary features can be extracted from an accurate tracing of the surface vascular network, but such tracings must still be done manually, with significant user intervention. Automating this procedure would not only allow more data acquisition to study the potential effects of placental health on later conditions, but may ideally serve as a real-time diagnostic for neonatal risk factors as well.

Much work has been to develop reliable vascular extraction methods for well-known image domains (such as retinal MRA images) using Hessian-based filters, namely the (multiscale) Frangi filter. It is desirable to extend these arguments to placental images, but this approach is greatly hindered by the inherent irregularity of the placental surface as a whole, which introduces significant noise into the image domain.  A recent attempt was made to apply an additional local curvilinear filter to the Frangi result in an effort to remove some noise from the final extraction.

Here we propose an alternate extraction method. First, we use arguments from Frangi's original paper to provide a proper selection of parameters for our particular image domain. Using the same arguments from differential geometry that gave rise to the Frangi filter, we calculate the leading principal direction (eigenvector of the Hessian) to indicate the directionality of curvilinear features at a particular scale. We are then able to apply an appropriately-oriented morphological filter to our Frangi targets at select scales to remove noise. This approach differs significantly from previous efforts in that morphological filtering will take place at each scale space, rather than being performed one time following multiscale synthesis. Noise removal performed in this way is expected to aide in coherent interpretation of targets that should appear in a connected network.

Finally, we discuss an important advancement in implementation--scale space conversion for differentiation (i.e. gaussian blur) via Fast Fourier Transform (FFT) rather than a more traditional convolution with a gaussian kernel, which offers a significant speedup.  This thesis will also contain a general, in depth summary of both multiscale Hessian filters and scale-space theory. 

We demonstrate the effectiveness of our improved vascular extraction technique on several of the following image domains: a private database of barium-injected samples provided by University of Rochester, uninjected/raw placental samples from Placental Analytics LLC, a collection of simulated images, the DRIVE and STARE databases of retinal MRAs, and a new collection of computer-generated images with significant curvilinear content.

Time permitting, this research will be extended to include a method of network connection, so that a logically connected vascular network is realized (i.e. network completion).
