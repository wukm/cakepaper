%%%%% Abstract

% University Guidelines suggest less than 150 words or 2 pages

Recent statistical analysis has suggested that accurate measurements of certain features of the placental chorionic surface vascular network (PCSVN) after birth may be useful in identifying risks of certain neurodevelopmental disorders, including Autism Spectrum Disorder. These features, taken from a single top-down photograph, can be extracted from a manual tracing of the PCSVN, but the tracing procedure itself is very time and resource intensive. Automating this tracing procedure would not only allow more data acquisition to study potential links between placental development and neonatal health, but even perhaps provide a real-time diagnostic for certain risk factors.

Much work has been to develop reliable vascular network segmentation methods for well-known image domains (such as retinal MRA images) using Hessian-based filters, namely the multiscale Frangi filter. It is desirable to extend these techniques to our image domain, but this approach has been historically hindered by the comparative irregularity of the placental surface as a whole, which introduces significant noise into filtered result. Here we provide an in-depth mathematical background of the multiscale Frangi filter, touching upon topics in differential geometry, linear scale space theory, multiscale methods, and morphological image processing. Informed by this theory, we are able to identify stricter parameters that allow us to greatly improve our result. We also reimplement the Frangi filter in frequency space (using a Fast Fourier Transform), which allows a major speedup compared to previous efforts.

Compared to recent success in solving this problem using more modern techniques such as neural networks, we demonstrate the continued viability of this classic image processing technique. Its comparative speed and simplicity make it suitable as a prefilter to more robust techniques. We apply a large (twenty scales) multiscale Frangi filter on a subset of 201 placental images from a private database provided by the National Children's Study (NCS). We then compare several approaches of merging the multiscale result into an approximation of the PCSVN and compare them to manual tracings of the network. Finally, we develop the notion of the signed Frangi filter, upon which we describe a straightforward segmentation method called "trough-filling" which is rather suitable for this particular image domain.

%Keywords: Medical Image Processing, Image Segmentation, Placenta, Differential Geometry, Multiscale Methods






