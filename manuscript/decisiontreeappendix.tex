We follow the procedure for \cref{ch:segmentation}. For our analysis of the Frangi filter apart from segmentation we shall stop before step (C).
    
    \begin{spacing}{1}
    \begin{small}
    \begin{verbatim}
For each sample:

A) Preprocessing
  1) Find placental border
  	a) Gradient of green channel for small sigma=.01
  	b) Watershed outside against above gradient mean
  	c) erode these points, radius=15
  	d) Identify largest object in watershed threshold
  2) RGB to single channel via Luminance Transform
  3) Glare removal
    a) Mask glare (threshold at *175/255*)
    b) Dilate glare mask with radius=2
    c) Inpaint glare
      + Hybrid inpainting, with size threshold *32*
      . Biharmonic inpainting
      . Median value of boundary
  4) Dilate around UCIP, add to mask (radius=50)
B) Multiscale Frangi filter
  1) Set parameters
    a) Scales
      = n_scales (default: *20*)
      = log_range (default *[-1.5, 3.2]*), log base 2
      -> scales
    b) Beta = 0.1, Gamma = 1.0 (or alternate parametrization)
    d) Dilate mask per scale (20 pixels)
  2) For each sigma: compute Uniscale Frangi Filter
    a) gauss blur image with discrete Gaussian kernel FFT
    b) take gradient across each axis of blurred image;
       take gradient across each axis of gradient
       -> (Hxx, Hxy, Hyy)
    c) Find eigenvalues of hessian at each point (using np.eig)
       and sort by absolute magnitude 
    d) Calculate Frangi Vesselness Measure
  3) Split positive and negative strain.
  3) Merge each result, reserve positive and negative stacks
C) Estimate PCSVN
  1) Approximate using strategy
    a) Calculate Vmax above high fixed threshold alpha=0.3
    b) Calculate Vmax above lower fixed threshold alpha=0.2
    b) Threshold scalewise at 95th nonzero-percentile
    c) Threshold scalewise at 98th nonzero-percentile
    d) Margin adding algorithm with high fixed threshold
  2) Compare to ground truth trace to obtain confusion matrix
  3) Calculate MCC score and precision
\end{verbatim}
    \end{small}
\end{spacing}