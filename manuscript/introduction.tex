\chapter{Introduction} \label{ch:introduction}

From \cite{chang2017}, it is useful to develop a neonatal test for high risk of
Autism Spectrum Disorder. There is some evidence as in \cite{chang2016whole}
of correlation between this risk and placental health factors.. Most ASD
cases are not diagnosed until the child reaches three or four, so the benefit
of an early detection technique would be massive, as the brain may be more
receptive to treatment at a young age. In particular, it was shown in
\cite{chang2016whole} that measurements of the placental chorionic surface
vascular network (PCSVN) may be useful in identifying such risk.
\cite{chang2017} has provided a method of automatically calculating such
features from an extracted vascular network, but does so with manual tracing of
the PCSVN in order to make these measurements.  These manual tracings are
labor-intensive, requiring 4 to 8 hours of user intervention to generate each trace.
The present work follows several other efforts towards an automated extraction
technique, such as \cite{almoussa-ucla-reu}, \cite{huynh2013filter}, and
most recently \cite{djima2017enhancing}. Automating this procedure would not
only allow more data acquisition to study the potential connection between ASD
and placental health, but could even potentially serve as the basis of real-time
diagnostic for neonatal risk factors as well. We continue the work of
developing a procedure to automate extraction of the PCSVN.


Our basic goal of "vascular network extraction" is a common one in image
processing. There are have been many techniques adapted to extracting vascular
networks. However, the placenta in particular poses as a particularly difficult
image domain to work with. It is a surface network with much irregularity: the are
frequent gaps and crossings, the "background" has a great degree of variation in
color intensity and structure itself, causing many na\"{i}ve
approaches to fail completely.

Reliable vascular extraction methods do exist for well-known image domains
(such as retinal MRA images) using Hessian-based filters, namely the (multiscale) Frangi filter.
It is desirable to extend this technique to study placental images, but this
approach is greatly hindered by the comparative irregularity of the placental
surface as a whole, which introduces significant noise into the image domain.
Previous attempts to implement the Frangi filter in this context \cite{huynh2013filter}
dealt with this noise by passing the Frangi filtered result to an additional local
curvilinear filter (a directional filter bank) in an effort to retrieve a partial
segmentation. Other more recent efforts \cite{djima2017enhancing} are very promising
but are considerably more resource-intensive.

Here we provide an in depth mathematical background of the Frangi filter and
its justification as an image-processing technique, as well as an introduction
to the Gaussian scale space theory common to many multiscale methods. Finally,
we discuss an important advancement in implementation--scale space conversion
for differentiation (i.e. gaussian blur) via Fast Fourier Transform, which
offers a significant speedup. This allows us faster calculation of the
eigenvalues of the Hessian, from which we calculate the Frangi filter, a
vesselness measure.

We demonstrate the effectiveness of our sped-up implementation of the Frangi
filter by performing a large ($N=20$) multiscale Frangi filter on a set of 201
placental images from a private database provided by the National Children's
Study (NCS). We then appeal to the usefulness of our multiscale output by
then demonstrating several approaches to determining a segmentation
of the PCSVN. These will range from naive (i.e direct thresholding) to
rather viable (scalewise random walks). We compare each of these results to
manual tracings of the network.
We shall our ability to take many more scales into consideration (by virtue of
the sped up Frangi filter) allows us to be pickier about our thresholding,
as well as our choice of parameters, which significantly reduces noise
experienced in previous efforts.