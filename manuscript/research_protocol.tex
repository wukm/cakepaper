\chapter{Research Protocol}

\vcomment{List all decisions you make. Be very explicit.
	Current plan: Go into depth for certain portions in the implementations chapter.
	This covers from 
List. All. Decisions. You Make. Be very explicit.}

\section{Image Preprocessing}
	\vcomment{Cover data cleaning in a different section/chapter dedicated to image domains.}
	All images are grayscale, $M,N$ pixels as a masked array, (background is manually designated as a mask).
	Covered to grayscale in the standard way.
	\vtodo{Note:  the only thing that makes sense to do here is maybe clipping. Any linear rescaling doesn't matter in terms of derivatives. Mention this somewhere? / Address the question of not needing to apply contrast enhancement.}
	
\section{Multiscale Setup}

	Pick a logarithmic range of scales $\Sigma := \{ \sigma_1, \sigma_2, \dots, \sigma_N\}$.
	The smallest one should be an effective size, largest should be an effective size as well. Designate these manually per image or use a default range and ``throw out'' bad ones somehow.
	
	Convolve this via fft transform to get $L_{\sigma_i}$
	
\section{Applying Vesselness Measure}
Calculate the Hessian matrix of  and then the eigenvalues using the function \texttt{hfft.fft\_hessian}.

\vtodo{Show at this point or move elsewhere how features of  size $\sigma$ are isolated in this step at the appropriate scale.}
\section{Scale-space post-processing}
\section{Multiscale Merging}
\section{Cleanup/Postprocessing}
\section{Measurements}
